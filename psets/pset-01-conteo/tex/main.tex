\title{EST241 Estad\'istica Inferencial}
\author{Manuel Loaiza Vasquez}
\date{Octubre 2021}

\begin{document}

\maketitle

\vspace*{-0.25in}
\centerline{Pontificia Universidad Cat\'olica del Per\'u}
\centerline{Lima, Per\'u}
\centerline{\mailto{manuel.loaiza@pucp.edu.pe}}
\vspace*{0.15in}

\begin{framed}
  Lista de ejercicios sobre t\'ecnicas de conteo para practicar antes del examen parcial.
\end{framed}

\section{Bolas y casilleros}

En esta secci\'on investigaremos los efectos de ciertos supuestos en el n\'umero de maneras
de colocar $n$ bolas en $b$ casillas.

\begin{statement}{1}
  Supongamos que las $n$ bolas son distintas y que el orden dentro de un casiller no importa.
  Pruebe que el n\'umero de maneras de colocar las bolas en los casilleros es $b^n$.
\end{statement}

\begin{statement}{2}
  Supongamos que las bolas son distintas y las bolas en cada casillero est\'an ordenadas.
  Pruebe que existen exactamente $(b + n + 1)! / (b - 1)!$ maneras de colocar las bolas en los casilleros.
\end{statement}

\begin{statement}{3}
  Supongamos que las bolas son id\'enticas, por lo que el orden de estas dentro de un casillero no importa.
  Pruebe que el n\'umero de maneras de colocar las bolas en los casilleros es $\binom{b + n - 1}{n}$.
\end{statement}

\begin{statement}{4}
  Construya un conjunto de $n$ eventos que son independientes dos a dos pero que
  ning\'un subconjunto de $k > 2$ de ellos es mutuamente independiente.
\end{statement}

\begin{statement}{5}
  Supongamos que las bolas son id\'enticas y ning\'un casillero termina vac\'io.
  Asumiendo $n \geq b$, pruebe que el n\'umero de maneras de colocar las bolas es $\binom{n - 1}{b - 1}$.
\end{statement}

\section{Conteo}

\begin{statement}{6}
  T\'u eres un participante de un show en el cual el premio est\'a oculto detr\'as de tres cortinas.
  T\'u ganas el premio si seleccionas la cortina correcta.
  Luego de que escojas una cortina pero antes de que se levante, el presentador levanta una de las
  otras cortinas, sabiendo que revelar\'a un escenario vac\'io, y te preguntar\'a si es que quieres
  cambiar la elecci\'on de tu cortina hacia la cortina restante.
  ¿C\'omo cambiar\'ian tus posibilidades si es que cambias de cortina?
\end{statement}

\begin{statement}{7}
  Una \textbf{funci\'on booleana} $n$-entrada, $m$-salida es una funci\'on
  \[
    f: \{0, 1\}^n \to \{0, 1\}^m.
  \]
  ¿Cu\'antas funciones booleanas $n$-entrada y $1$-salida existen?
  ¿Cu\'antas funciones booleanas $n$-entrada y $m$-salida existen?
\end{statement}

\begin{statement}{8}
  Supongamos que uno tira dos dados ordinarios de seis caras.
  ¿Cu\'al es el valor esperado de la suma de los dos valores obtenidos?
  ¿Cu\'al es el valor esperado del m\'aximo de los dos valores obtenidos?
\end{statement}

\begin{statement}{9}
  Un arreglo $A[1 \dots n]$ contiene $n$ n\'umeros distintos que est\'an aleatoriamente ordenados,
  con cada permutaci\'on de los $n$ nu\'meros siendo equiprobable.
  ¿Cu\'al es el valor esperado del \'indice del mayor elemento en el arreglo?
  ¿Cu\'al es el valor esperado del \'indice del menor elemento en el arreglo?
\end{statement}
    
\begin{statement}{10}
  Pruebe la identidad
  \[
    \binom{n}{k} = \frac{n}{k} \binom{n - 1}{k - 1}
  \]
  para $0 < k \leq n$.
\end{statement}

\begin{statement}{11}
  Pruebe la identidad
  \[
    \binom{n}{k} = \frac{n}{n - k} \binom{n - 1}{k}
  \]
  para $0 \leq k < n$.
\end{statement}

\begin{statement}{12}
  Sean $X$ e $Y$ variables aleatorias independientes.
  Pruebe que $f(x)$ y $g(Y)$ son independientes para cualquier elecci\'on de funciones $f$ y $g$.
\end{statement}

\begin{statement}{13}
  Sea $X$ una variable aleatoria no negativa y supongamos que $\BE[X]$ est\'a bien definido.
  Pruebe la \textbf{desigualdad de Markov}:
  \[
    \BP[X \geq t] \leq \frac{\BE[X]}{t}
  \]
  para todo $t > 0$.
\end{statement}

\begin{statement}{14}
  Pruebe que cualquier par de enteros $n \geq 0$ y $0 \leq k \leq n$, la expresi\'on
  $\binom{n}{k}$ alcanza su m\'aximo cuando $k = \floor{n / 2}$ o $k = \ceiling{n / 2}$.
\end{statement}

\begin{statement}{15}
  Pruebe que para cualquier conjunto de enteros $n \geq 0$, j $\geq 0$, $k \geq 0$ y $j + k \leq n$,
  \[
    \binom{n}{j + k}  \leq \binom{n}{j} \binom{n - j}{k}.
  \]
  Provea una prueba algebraica y una prueba utilizando un argumento combinatorio.
\end{statement}

\begin{statement}{16}
  Utilice inducci\'on para todos los enteros $k$ tales que $0 \leq k \leq n / 2$ para probar
  \[
    \binom{n}{k} = \frac{n^n}{k^k (n - k)^{n - k}}
  \]
  y exti\'endalo a todos los enteros $k$ tales que $0 \leq k \leq n$.
\end{statement}

\begin{statement}{17}
  Pruebe que $\Var[aX] = a^2 \Var[X]$.
\end{statement}

\begin{statement}{18}
  Sea $n \geq 0$, $0 < p < 1$, $q = 1 - p$ y $0 \geq k \geq n$. Luego
  \[
    \CB(k; n, p) \geq \binom{np}{k}^k \binom{nq}{n - k}^{n - k}.
  \]
\end{statement}

\begin{statement}{19}
  Pruebe que
  \[
    \binom{n}{2} = \binom{k}{2} + k (n - k) + \binom{n - k}{2}
  \]
  con $1 \leq k \leq n$.
\end{statement}

\begin{statement}{20}
  Pruebe que el valor del m\'aximo de la distribuci\'on binomial $\CB(k; n, p)$ es
  aproximadamente $1 / \sqrt{2 \pi n p q}$, con $q = 1 - p$.
\end{statement}

\begin{statement}{21}
  Contar el n\'umero de soluciones de la siguiente ecuaci\'on
  \[
    x_1 + x_2 + \cdots + x_k = n,  
  \]
  con $x_i \in \BN \cup \{0\},\, i = 1, \dots, k$.
\end{statement}

\begin{statement}{22}
  Considere $\sigma: \{1, 2, \dots, n\} \to \{1, 2, \dots, n\}$ las funciones
  permutaci\'on.
  \begin{itemize}
    \item ¿Cu\'antas $\sigma$ tienen \'unicamente un ciclo? Es decir, si
    tenemos $\sigma(1), \sigma\, \circ\, \sigma(1), \sigma\, \circ\, \sigma\, \circ\,
    \sigma(1), \dots$ habremos iterado sobre todos los elementos
    $\{1, 2, \dots, n\}$.
    \item ¿Cu\'antas $\sigma$ no tienen puntos fijos? Es decir, tienen la
    propiedad de que para cada $i$, $\sigma(i) \neq i$.
    \item ¿Cu\'antas $\sigma$ son involuciones sin puntos fijos? Es decir, tienen
    la propiedad de que para cada $i$, $\sigma(i) \neq i$ pero $\sigma \circ \sigma(i) = i$.
  \end{itemize}
\end{statement}

\begin{statement}{23}
  Un torneo todos contra todos de $n$ participantes es un torneo en el cual
  cada una de las $\binom{n}{2}$ parejas de participantes juega uno contra el otro
  exactamente una vez, con un resultado de cualquier juego obteniendo un
  participante ganador y otro perdedor.
  Sea $k$ un entero fijo, $k < n$, una pregunta que nos puede interesar es si
  es que es posible que el resultado del torneo sea tal que, para todo conjunto
  de $k$ jugadores, existe un jugador que puede vencer a cada integrante de ese
  conjunto. Pruebe que si
  \[
    \binom{n}{k}\left[1 - \left(\frac{1}{2}\right)^k\right]^{n - k} < 1
  \]
  entonces dicho resultado es posible.
\end{statement}

\begin{statement}{24}
  Dados dos n\'umeros naturales $n$ y $k$. Hallar la m\'axima potencia de $k$
  que divide a $n!$.
\end{statement}

\begin{statement}{25}
  Sea $X$ una variable aleatoria para el n\'umero total de \'exitos en un conjunto $A$ de $n$ experimentos
  de Bernoulli, donde el $i$-\'esimo de los experimentos tiene probabilidad $p_i$ de \'exito, y sea $X'$
  una variable aleatoria para el total de n\'umeros de \'exitos en un segundo conjunto $A'$ de $n$
  experimentos de Bernoulli, donde el $i$-\'esimo experimento tiene probabilidad $p_i' \geq p_i$ de \'exito.
  Pruebe que para todo $0 \leq k \leq n$,
  \[
    \BP[X' \geq k] \geq \BP[X \geq k].
  \]
\end{statement}

\begin{statement}{26}
  ¿Cu\'al es menos probable: no obtener caras cuando lanzas una moneda $n$ veces, u
  obtener menos de $n$ caras cuando tiras una moneda $4n$ veces?
\end{statement}

\begin{statement}{27}
  Lijie Chen decide visitar la granja de su t\'io Ce Jin.
  En la granja hay $s$ animales y $n$ corrales.
  Por practicidad, los corrales se construyeron en una sola fila.
  Ce Jin le cont\'o a Lijie Chen que una distribuci\'on de la granja es \textit{suertuda} si
  todos los animales se encuentran en todos los corrales de tal manera de que ning\'un corral quede vac\'io
  y que exista al menos un segmento continuo de corrales con exactamente $k$ animales en total.
  Es m\'as, una granja es \textit{ideal} si es \textit{suertuda} para toda distribuci\'on
  sin corrales vaci\'os.
  Ni Ce Jin ni Lijie Chen saben si la granja es ideal o no.
  Dados $n \leq s$ y $k$ enteros positivos ¿podemos ayudarles a determinar esto?
\end{statement}

Definimos la \textbf{funci\'on de entrop\'ia} a $H: [0, 1] \to \BR$ con
\[
    H(\lambda)     = -\lambda \log_2 \lambda - (1 - \lambda) \log_2 (1 - \lambda)
\]
donde, por conveniencia, asumimos que $0 \log_2 0 = 0$, por lo que $H(0) = H(1) = 0$.

\begin{statement}{28}
  Diferenciando la funci\'on de entrop\'ia $H$, pruebe que alcanza su m\'aximo valor en
  $\lambda = 1 / 2$. ¿Qu\'e valor tiene $H(1 / 2)$?
\end{statement}

\begin{statement}{29}
  Pruebe que si $0 < k < np$, donde $0 < p < 1$ y $q = 1 - p$, luego
  \[
    \sum_{i = 0}^{k - 1} p^i q^{n - i} < \frac{kq}{np - k}
    \left(\frac{np}{k}\right)^k
    \left(\frac{nq}{n - k}\right)^{n - k}.
  \]
\end{statement}

\section{Aproximaci\'on de Stirling}

Una cota superior un poco d\'ebil de la funci\'on factorial es $n! \leq n^n$, pues cada
uno de los $n$ t\'erminos en el producto factorial es a lo m\'as $n$.
La \textbf{aproximaci\'on de Stirling}
\[
    n! = \sqrt{2 \pi n} \left(\frac{n}{e}\right)^n \left(1 + \Theta\left(\frac{1}{n}\right)\right)
\]
nos da una cota superior m\'as apretada, y una cota inferior tambi\'en.

\begin{theorem}
    Dado un n\'umero entero positivo $n$, se cumple $n! = o(n^n)$.
\end{theorem}

\begin{proof}
    Ejercicio para el lector.
\end{proof}

\begin{theorem}
    Dado un n\'umero entero positivo $n$, se cumple $n! = \omega(2^n)$.
\end{theorem}

\begin{proof}
    Ejercicio para el lector.
\end{proof}

\begin{theorem}
    Dado un n\'umero entero positivo $n$, se cumple $\log_2(n!) = \Theta(n \log_2 n)$.
\end{theorem}

\begin{proof}
    Ejercicio para el lector.
\end{proof}

\begin{theorem}
    La siguiente ecuaci\'on se cumple para $n \geq 1$
    \[
        n! = \sqrt{2 \pi n} \left(\frac{n}{e}\right)^n e^{\alpha_n}
    \]
    donde
    \[
        \frac{1}{12n + 1} < \alpha_n < \frac{1}{12n}.
    \]
\end{theorem}

\begin{proof}
    Ejercicio para el lector.
\end{proof}

\begin{statement}{30}
  Pruebe que
  \[
    \BP[\mu - X \geq r] \leq \left(\frac{(n - \mu) e}{r}\right)^r
  \]
  para $r > n - \mu$. Similarmente, pruebe que
  \[
    \BP[np - X \geq r] \leq \left(\frac{nqe}{r}\right)^r
  \]
  para $r > n - np$.
\end{statement}

\end{document}