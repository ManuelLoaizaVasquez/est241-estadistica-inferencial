\title{EST241 Estad\'istica Inferencial}
\author{Manuel Loaiza Vasquez}
\date{Octubre 2021}

\begin{document}

\maketitle

\vspace*{-0.25in}
\centerline{Pontificia Universidad Cat\'olica del Per\'u}
\centerline{Lima, Per\'u}
\centerline{\mailto{manuel.loaiza@pucp.edu.pe}}
\vspace*{0.15in}

\begin{framed}
  Lista de ejercicios sobre t\'ecnicas de conteo para practicar antes del examen parcial.
\end{framed}

\section{Bolas y casilleros}

En esta secci\'on investigaremos los efectos de ciertos supuestos en el n\'umero de maneras
de colocar $n$ bolas en $b$ casillas.

\begin{statement}{1}
  Pruebe que para cualquier cantidad enumerable de eventos $(A_n)_{n \in \BN}$
  \[
    \BP\left[\bigcup_{n \in \BN} A_n\right] \leq \sum_{n \in \BN} \BP[A_n].
  \]
\end{statement}

\begin{statement}{2}
  Supongamos que las bolas son distintas y las bolas en cada casillero est\'an ordenadas.
  Pruebe que existen exactamente $(b + n + 1)! / (b - 1)!$ maneras de colocar las bolas en los casilleros.
\end{statement}

\begin{statement}{3}
  Describa un algoritmo que tome como entrada dos n\'umeros enteros $a$ y $b$ tales que
  $0 < a < b$ y, lanzando una moneda con resultados equiprobables, produce como salida
  caras con probabilidad $a / b$ y sellos con probabilidad $(b - a) / b$.
  Brinde una cota para el valor esperado de lanzamiento de monedas, el cu\'al debe ser $O(1)$.
\end{statement}

\begin{statement}{4}
  Supongamos que las bolas son id\'enticas y ning\'un casillero contiene m\'as de una bola,
  por lo que $n \leq b$.
  Pruebe que el n\'umero de maneras de colocar las bolas es $\binom{b}{n}$.
\end{statement}

\begin{statement}{5}
  Supongamos que las bolas son id\'enticas y ning\'un casillero termina vac\'io.
  Asumiendo $n \geq b$, pruebe que el n\'umero de maneras de colocar las bolas es $\binom{n - 1}{b - 1}$.
\end{statement}

\section{Conteo}

\begin{statement}{6}
  ¿Cu\'antas $k$-subcadenas tiene una $n$-cadena?
  ¿Cua\'antas subcadenas tiene una $n$-cadena en total?
\end{statement}

\begin{statement}{7}
  Una \textbf{funci\'on booleana} $n$-entrada, $m$-salida es una funci\'on
  \[
    f: \{0, 1\}^n \to \{0, 1\}^m.
  \]
  ¿Cu\'antas funciones booleanas $n$-entrada y $1$-salida existen?
  ¿Cu\'antas funciones booleanas $n$-entrada y $m$-salida existen?
\end{statement}

\begin{statement}{8}
  ¿De cu\'antas maneras se pueden sentar $n$ personas alrededor de una mesa circular?
  Consideremos que dos configuraciones son iguales si es que una puede ser obtenida rotando la otra.
\end{statement}

\begin{statement}{9}
  ¿De cu\'antas maneras podemos escoger tres n\'umeros distintos del conjunto $\{1, 2, \dots, 99\}$
  tal que su suma sea par?
\end{statement}
    
\begin{statement}{10}
  Pruebe la identidad
  \[
    \binom{n}{k} = \frac{n}{k} \binom{n - 1}{k - 1}
  \]
  para $0 < k \leq n$.
\end{statement}

\begin{statement}{11}
  Pruebe que si $X$ e $Y$ son variables aleatorias no negativas, luego
  \[
    \BE[\max(X, Y)] \leq \BE[X] + \BE[Y].
  \]
\end{statement}

\begin{statement}{12}
  Sean $X$ e $Y$ variables aleatorias independientes.
  Pruebe que $f(x)$ y $g(Y)$ son independientes para cualquier elecci\'on de funciones $f$ y $g$.
\end{statement}

\begin{statement}{13}
  Sea $X$ una variable aleatoria no negativa y supongamos que $\BE[X]$ est\'a bien definido.
  Pruebe la \textbf{desigualdad de Markov}:
  \[
    \BP[X \geq t] \leq \frac{\BE[X]}{t}
  \]
  para todo $t > 0$.
\end{statement}

\begin{statement}{14}
  Pruebe que cualquier par de enteros $n \geq 0$ y $0 \leq k \leq n$, la expresi\'on
  $\binom{n}{k}$ alcanza su m\'aximo cuando $k = \floor{n / 2}$ o $k = \ceiling{n / 2}$.
\end{statement}

\begin{statement}{15}
  ¿Cu\'al es m\'as grande, el valor esperado del cuadrado de una variable aleatoria,
  o el cuadrado de su valor esperado?
\end{statement}

\begin{statement}{16}
  Utilice inducci\'on para todos los enteros $k$ tales que $0 \leq k \leq n / 2$ para probar
  \[
    \binom{n}{k} = \frac{n^n}{k^k (n - k)^{n - k}}
  \]
  y exti\'endalo a todos los enteros $k$ tales que $0 \leq k \leq n$.
\end{statement}

\begin{statement}{17}
  Pruebe que $\Var[aX] = a^2 \Var[X]$.
\end{statement}

\begin{statement}{18}
  Pruebe la identidad
  \[
    \binom{n}{k} = \sum_{i = k}^n \binom{i - 1}{k - 1}
  \]
  para $n \geq k$.
\end{statement}

\begin{statement}{19}
  Pruebe que $\CB(k; n, p) = \CB(n - k; n, q)$, donde $q = 1 - p$.
\end{statement}

\begin{statement}{20}
  Pruebe que
  \[
    \sum_{k = 0}^n \binom{n}{k} 2^k = 3^n.  
  \]
\end{statement}

\begin{statement}{21}
  Pruebe que la probabilidad de no fracasar en $n$ experimentos de Bernoulli, cada uno con probabilidad
  $p = 1 / n$, es aproximadamente $1 / e$. Pruebe que la probabilidad de que exactamente ocurra un
  \'exito es tambi\'en aproximadamente $1 / e$.
\end{statement}

\begin{statement}{22}
  Yinzhan Xu lanza una moneda $n$ veces, y tambi\'en Yuhao Du.
  Pruebe que la probabilidad de que ellos tengan el mismo n\'umero de caras es
  $\binom{2n}{n}/4^n$.
  Use su argumento para verificar la identidad
  \[
    \sum_{k = 0}^n \binom{n}{k}^2 = \binom{2n}{n}.
  \]
\end{statement}

\begin{statement}{23}
  Pruebe que para $0 \leq k \leq n$,
  \[
    \CB(k; n, 1/2) \leq 2^{n H(k / n) - n}.
  \]
\end{statement}

\begin{statement}{24}
  Considere $n$ experimentos de Bernoulli, donde el $i$-\'esimo experimento tiene
  probabilidad $p_i$ de \'exito con $i = 1, \dots, n$, y sea $X$ una variable aleatoria denotando
  el n\'umero total de \'exitos. Sea $p \geq p_i$, para todo $i = 1, \dots, n$.
  Pruebe que para $1 \leq k \leq n$,
  \[
    \BP[X < k] \geq \sum_{i = 0}^{k - 1} \CB(i; n, p).
  \]
\end{statement}

\begin{statement}{25}
  Calcule el n\'umero de maneras de colocar $k$ alfiles en un tablero de ajedrez
  de $n \times n$ tal que ning\'un par de alfil se puede atacar mutuamente.
\end{statement}

\begin{statement}{26}
  Contar el n\'umero de maneras de conectar $2n$ puntos en una circunferencia
  para formar $n$ cuerdas disjuntas.
\end{statement}

\begin{statement}{27}
  Lijie Chen decide visitar la granja de su t\'io Ce Jin.
  En la granja hay $s$ animales y $n$ corrales.
  Por practicidad, los corrales se construyeron en una sola fila.
  Ce Jin le cont\'o a Lijie Chen que una distribuci\'on de la granja es \textit{suertuda} si
  todos los animales se encuentran en todos los corrales de tal manera de que ning\'un corral quede vac\'io
  y que exista al menos un segmento continuo de corrales con exactamente $k$ animales en total.
  Es m\'as, una granja es \textit{ideal} si es \textit{suertuda} para toda distribuci\'on
  sin corrales vaci\'os.
  Ni Ce Jin ni Lijie Chen saben si la granja es ideal o no.
  Dados $n \leq s$ y $k$ enteros positivos ¿podemos ayudarles a determinar esto?
\end{statement}

Definimos la \textbf{funci\'on de entrop\'ia} a $H: [0, 1] \to \BR$ con
\[
    H(\lambda)     = -\lambda \log_2 \lambda - (1 - \lambda) \log_2 (1 - \lambda)
\]
donde, por conveniencia, asumimos que $0 \log_2 0 = 0$, por lo que $H(0) = H(1) = 0$.

\begin{statement}{28}
  Pruebe que
  \[
    \sum_{i = 0}^{k - 1} \binom{n}{i} a^i < (a + 1)^n \frac{k}{na - k(a + 1)} \CB(k; n, a / (a + 1))
  \]
  para todo $a > 0$ y todo $0 < k < na/(a + 1)$.
\end{statement}

\begin{statement}{29}
  Pruebe que si $0 < k < np$, donde $0 < p < 1$ y $q = 1 - p$, luego
  \[
    \sum_{i = 0}^{k - 1} p^i q^{n - i} < \frac{kq}{np - k}
    \left(\frac{np}{k}\right)^k
    \left(\frac{nq}{n - k}\right)^{n - k}.
  \]
\end{statement}

\section{Aproximaci\'on de Stirling}

Una cota superior un poco d\'ebil de la funci\'on factorial es $n! \leq n^n$, pues cada
uno de los $n$ t\'erminos en el producto factorial es a lo m\'as $n$.
La \textbf{aproximaci\'on de Stirling}
\[
    n! = \sqrt{2 \pi n} \left(\frac{n}{e}\right)^n \left(1 + \Theta\left(\frac{1}{n}\right)\right)
\]
nos da una cota superior m\'as apretada, y una cota inferior tambi\'en.

\begin{theorem}
    Dado un n\'umero entero positivo $n$, se cumple $n! = o(n^n)$.
\end{theorem}

\begin{proof}
    Ejercicio para el lector.
\end{proof}

\begin{theorem}
    Dado un n\'umero entero positivo $n$, se cumple $n! = \omega(2^n)$.
\end{theorem}

\begin{proof}
    Ejercicio para el lector.
\end{proof}

\begin{theorem}
    Dado un n\'umero entero positivo $n$, se cumple $\log_2(n!) = \Theta(n \log_2 n)$.
\end{theorem}

\begin{proof}
    Ejercicio para el lector.
\end{proof}

\begin{theorem}
    La siguiente ecuaci\'on se cumple para $n \geq 1$
    \[
        n! = \sqrt{2 \pi n} \left(\frac{n}{e}\right)^n e^{\alpha_n}
    \]
    donde
    \[
        \frac{1}{12n + 1} < \alpha_n < \frac{1}{12n}.
    \]
\end{theorem}

\begin{proof}
    Ejercicio para el lector.
\end{proof}

\begin{statement}{30}
  Pruebe que
  \[
    \binom{2n}{n} = \frac{2^{2n}}{\sqrt{\pi n}} (1 + O(1 / n)).
  \]
\end{statement}

\end{document}