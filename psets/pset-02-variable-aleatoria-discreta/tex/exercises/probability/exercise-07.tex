\begin{statement}{7}
  El guardi\'an de una prisi\'on ha escogido aleatoriamente un prisionero de tres para liberarlo.
  Los otros dos ser\'an ejecutados.
  El guardi\'an sabe qui\'en ser\'a liberado pero est\'a prohibido dar informaci\'on a los priosioneros
  acerca de sus estados.
  Nombremos a los priosioneros $X$, $Y$ y $Z$.
  El prisionero $X$ le pregunta al guardi\'an en privado cu\'al de $Y$ o $Z$ ser\'a ejecutado,
  argumentando que como \'el ya sabe que uno de los dos morir\'a, el guardia no revelar\'a informaci\'on sobre \'este.
  El guardi\'an le dice a $X$ que $Y$ ser\'a ejecutado.
  El prisionero $X$ se siente feliz ahora, pues se da cuenta que o bien \'el o bien el prisionero $Z$ ser\'an
  liberados, lo cual significa que su probabilidad de ser liberado es de $1 / 2$.
  ¿Est\'a \'el en lo correcto o sus probabilidades siguen siendo $1 / 3$?
\end{statement}