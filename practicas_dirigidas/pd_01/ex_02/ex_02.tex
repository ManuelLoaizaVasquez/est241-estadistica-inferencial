\documentclass{article}
\usepackage[utf8]{inputenc}
\usepackage{amsfonts,latexsym,amsthm,amssymb,amsmath,amscd,euscript}
\usepackage{mathtools}
\usepackage{framed}
% Descomentar fullpage cuando se quiera utilizar menos margen horizontal
%\usepackage{fullpage}
\usepackage{hyperref}
    \hypersetup{colorlinks=true,citecolor=blue,urlcolor =black,linkbordercolor={1 0 0}}

\newenvironment{statement}[1]{\smallskip\noindent\color[rgb]{1.00,0.00,0.50} {\bf #1.}}{}
\allowdisplaybreaks[1]

% Comandos para teoremas, definiciones, ejemplos, lemas, etc. para sus respectivos body types.
\renewcommand*{\proofname}{Prueba}
\renewcommand{\contentsname}{Contenido}

\newtheorem{theorem}{Teorema}
\newtheorem*{proposition}{Proposici\'on}
\newtheorem{lemma}[theorem]{Lema}
\newtheorem{corollary}[theorem]{Corolario}
\newtheorem{conjecture}[theorem]{Conjetura}
\newtheorem*{postulate}{Postulado}
\theoremstyle{definition}
\newtheorem{defn}[theorem]{Definici\'on}
\newtheorem{example}[theorem]{Ejemplo}

\theoremstyle{remark}
\newtheorem*{remark}{Observaci\'on}
\newtheorem*{notation}{Notaci\'on}
\newtheorem*{note}{Nota}

% Define tus comandos para hacer la vida más fácil.
\newcommand{\BR}{\mathbb R}
\newcommand{\BC}{\mathbb C}
\newcommand{\BF}{\mathbb F}
\newcommand{\BQ}{\mathbb Q}
\newcommand{\BZ}{\mathbb Z}
\newcommand{\BN}{\mathbb N}

\title{EST241 Estad\'istica Inferencial}
\author{Manuel Loaiza Vasquez}
\date{Septiembre 2021}

\begin{document}

\maketitle

\vspace*{-0.25in}
\centerline{Pontificia Universidad Cat\'olica del Per\'u}
\centerline{Lima, Per\'u}
\centerline{\href{mailto:manuel.loaiza@pucp.edu.pe}{{\tt manuel.loaiza@pucp.edu.pe}}}
\vspace*{0.15in}

\begin{framed}
  Soluci\'on del ejercicio $2$ de la pr\'actica dirigida $1$ del curso
  Estad\'istica Inferencial dictado por el profesor Luis Valdivieso Serrano
  y el jefe de pr\'actica Gerald Lozano.
\end{framed}

\begin{statement}{2}
    Un examen consta de $10$ preguntas de opci\'on m\'ultiple con $5$
    alternativas por pregunta.
    Cada pregunta bien constestada vale $2$ puntos y $0$ en caso contrario.
    Si un alumno marca al azar y sus respuestas son independientes.
\end{statement}

\begin{statement}{a}
  ¿Con qu\'e probabilidad aprobar\'a el examen?
\end{statement}

\begin{proof}
  Definamos \'exito como responder correctamente una pregunta y fracaso
  si ocurre lo contrario. Como el alumno marca al azar, tenemos
  $P(\text{\'exito}) = 1 / 5$ y $P(\text{fracaso}) = 4 / 5$.
  Definimos la variable aleatoria
  \[
    X = \text{n\'umero de respuestas correctas en $n$ preguntas contestadas}.
  \]
  Tenemos que $X \sim B(10, 1 / 5)$.

  Luego, para aprobar el examen necesitamos m\'as de la mitad de preguntas
\end{proof}

\begin{statement}{b}
  ¿Qu\'e nota se esperar\'a que obtenga y cu\'al ser\'ia la desviaci\'on
  est\'andar de su nota?
\end{statement}

\begin{statement}{c}
  Si cambian las reglas descontando ahora $0.25$ puntos por pregunta mal
  constestada o no contestada, ¿c\'omo cambian (a) y (b)?
  ¿Habr\'ia ahora mayor dispersi\'on en la nota que este alumno podr\'ia obtener?
\end{statement}

\begin{statement}{d}
  Si el tiempo que le demora a un alumno responder cada pregunta se asume que
  tiene una distribuci\'on normal de media $10$ minutos con desviaci\'on
  est\'andar de $2$ miuntos.
  ¿Con qu\'e probabilidad un alumno se demorar\'a en responder una pregunta
  m\'as de $4$ minutos?
  Piense ahora en una pregunta dif\'icil que no sigue el patr\'on anterior, pero
  que si tiene la misma dispersi\'on.
  Si el $90\%$ de los alumnos se ha demorado m\'as de $8$ minutos en responderla,
  ¿cu\'al ser\'ia el tiempo medio de respuesta de esa pregunta?
\end{statement}

\end{document}