\documentclass{article}
\usepackage[utf8]{inputenc}
\usepackage{amsfonts,latexsym,amsthm,amssymb,amsmath,amscd,euscript}
\usepackage{mathtools}
\usepackage{framed}
% Descomentar fullpage cuando se quiera utilizar menos margen horizontal
%\usepackage{fullpage}
\usepackage{hyperref}
    \hypersetup{colorlinks=true,citecolor=blue,urlcolor =black,linkbordercolor={1 0 0}}

\newenvironment{statement}[1]{\smallskip\noindent\color[rgb]{1.00,0.00,0.50} {\bf #1.}}{}
\allowdisplaybreaks[1]

% Comandos para teoremas, definiciones, ejemplos, lemas, etc. para sus respectivos body types.
\renewcommand*{\proofname}{Prueba}
\renewcommand{\contentsname}{Contenido}

\newtheorem{theorem}{Teorema}
\newtheorem*{proposition}{Proposici\'on}
\newtheorem{lemma}[theorem]{Lema}
\newtheorem{corollary}[theorem]{Corolario}
\newtheorem{conjecture}[theorem]{Conjetura}
\newtheorem*{postulate}{Postulado}
\theoremstyle{definition}
\newtheorem{defn}[theorem]{Definici\'on}
\newtheorem{example}[theorem]{Ejemplo}

\theoremstyle{remark}
\newtheorem*{remark}{Observaci\'on}
\newtheorem*{notation}{Notaci\'on}
\newtheorem*{note}{Nota}

% Define tus comandos para hacer la vida más fácil.
\newcommand{\BR}{\mathbb R}
\newcommand{\BC}{\mathbb C}
\newcommand{\BF}{\mathbb F}
\newcommand{\BQ}{\mathbb Q}
\newcommand{\BZ}{\mathbb Z}
\newcommand{\BN}{\mathbb N}

\title{EST241 Estad\'istica Inferencial}
\author{Manuel Loaiza Vasquez}
\date{Septiembre 2021}

\begin{document}

\maketitle

\vspace*{-0.25in}
\centerline{Pontificia Universidad Cat\'olica del Per\'u}
\centerline{Lima, Per\'u}
\centerline{\href{mailto:manuel.loaiza@pucp.edu.pe}{{\tt manuel.loaiza@pucp.edu.pe}}}
\vspace*{0.15in}

\begin{framed}
  Soluci\'on del ejercicio $14$ de la pr\'actica dirigida $1$ del curso
  Estad\'istica Inferencial dictado por el profesor Luis Valdivieso Serrano
  y el jefe de pr\'actica Gerald Lozano.
\end{framed}

\begin{statement}{a}
  Si $X \sim \text{Exp}(\beta)$ y $\mathbb{P}(X > 5\, |\, X > 3) = e^{-2}$.
  Halle $\beta$.
\end{statement}

\begin{statement}{b}
  Si $f_X(x) = 2 - a x^2$, para $|x| \leq 8$, halle el valor de $a$ para que esta
  sea una funci\'on de densidad de la variable aleatoria $X$ y calcule
  $\mathbb{P}(X > 5\, |\, X > 3)$.
\end{statement}

\begin{statement}{c}
  Si $X \sim \text{Exp}(\beta)$, halle la funci\'on de densidad de la variable
  aleatoria $Y = X^2$ y obtenga tambi\'en el valor esperado y la varianza
  de esta variable aleatoria.
\end{statement}

\begin{statement}{d}
  Si $X \sim \text{Exp}(\beta)$, halle su funci\'on de generadora de momentos y
  utilice esta para obtener su coeficiente de sesgo de Pearson, el cual est\'a
  definido por
  \[
    \mathbb{E}\left[\frac{(X - \mathbb{E}[X])^3}{\sigma_X^3}\right],
  \]
  donde $\sigma_X$ denota a la desviaci\'on est\'andar de la variable aleatoria $X$.
\end{statement}

\begin{statement}{e}
  Si $X \sim \mathcal{U}_{[-q, q]}$ con $q > 0$. Halle el valor de que $q$ de
  modo que
  \[
    \mathbb{P}(|X| < 2) = \mathbb{P}(|X| > 2).  
  \]
\end{statement}

\begin{statement}{f}
  Si $X$ tiene una distruci\'on uniforme $\mathcal{U}_{[0, 9]}$, obtenga la
  funci\'on de densidad, valor esperado y desviaci\'on est\'andar de la
  variable aleatoria $Y = \sqrt{X}$.
\end{statement}

\end{document}