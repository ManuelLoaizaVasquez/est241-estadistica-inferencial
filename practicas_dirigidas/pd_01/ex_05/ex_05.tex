\documentclass{article}
\usepackage[utf8]{inputenc}
\usepackage{amsfonts,latexsym,amsthm,amssymb,amsmath,amscd,euscript}
\usepackage{mathtools}
\usepackage{framed}
% Descomentar fullpage cuando se quiera utilizar menos margen horizontal
%\usepackage{fullpage}
\usepackage{hyperref}
    \hypersetup{colorlinks=true,citecolor=blue,urlcolor =black,linkbordercolor={1 0 0}}

\newenvironment{statement}[1]{\smallskip\noindent\color[rgb]{1.00,0.00,0.50} {\bf #1.}}{}
\allowdisplaybreaks[1]

% Comandos para teoremas, definiciones, ejemplos, lemas, etc. para sus respectivos body types.
\renewcommand*{\proofname}{Prueba}
\renewcommand{\contentsname}{Contenido}

\newtheorem{theorem}{Teorema}
\newtheorem*{proposition}{Proposici\'on}
\newtheorem{lemma}[theorem]{Lema}
\newtheorem{corollary}[theorem]{Corolario}
\newtheorem{conjecture}[theorem]{Conjetura}
\newtheorem*{postulate}{Postulado}
\theoremstyle{definition}
\newtheorem{defn}[theorem]{Definici\'on}
\newtheorem{example}[theorem]{Ejemplo}

\theoremstyle{remark}
\newtheorem*{remark}{Observaci\'on}
\newtheorem*{notation}{Notaci\'on}
\newtheorem*{note}{Nota}

% Define tus comandos para hacer la vida más fácil.
\newcommand{\BR}{\mathbb R}
\newcommand{\BC}{\mathbb C}
\newcommand{\BF}{\mathbb F}
\newcommand{\BQ}{\mathbb Q}
\newcommand{\BZ}{\mathbb Z}
\newcommand{\BN}{\mathbb N}

\title{EST241 Estad\'istica Inferencial}
\author{Manuel Loaiza Vasquez}
\date{Septiembre 2021}

\begin{document}

\maketitle

\vspace*{-0.25in}
\centerline{Pontificia Universidad Cat\'olica del Per\'u}
\centerline{Lima, Per\'u}
\centerline{\href{mailto:manuel.loaiza@pucp.edu.pe}{{\tt manuel.loaiza@pucp.edu.pe}}}
\vspace*{0.15in}

\begin{framed}
  Soluci\'on del ejercicio $5$ de la pr\'actica dirigida $1$ del curso
  Estad\'istica Inferencial dictado por el profesor Luis Valdivieso Serrano
  y el jefe de pr\'actica Gerald Lozano.
\end{framed}

\begin{statement}{5}
  Una acci\'on valuada incialmente en S$/. 10$, se cotiza en dos periodos.
  Durante cada periodo la acci\'on puede bajar, permanecer igual o subir en S$/. 1$.
  Las probabilidades de estos eventos en el primer periodo son $0.2$, $0.5$ y $0.3$ respectivamente.
  En el segundo, la acci\'on puede volver a subir con probabilidad $0.3$ y bajar
  despu\'es de subir en el primero con probabilidad $0.1$.
  Ahora, si la acci\'on se mantiene igual en el periodo inicial, puede subir en
  el segundo con probabilidad $0.4$ y puede mantenerse igual con probabilidad
  $0.35$. Finalmente, si la acci\'on baja en el primer periodo, lo seguir\'a
  haciendo en el segundo periodo con probabilidad $0.5$, y subir\'a con
  probabilidad $0.1$.
\end{statement}

\begin{statement}{a}
  Halle la probabilidad de que el valor final de la acci\'on sea de S$/. 12$.
\end{statement}

\begin{statement}{b}
  Si el valor final de la acci\'on iguala al valor inicial, ¿cu\'al ser\'ia
  la probabilidad de que la acci\'on haya bajado?
\end{statement}

\begin{statement}{c}
  Sea $X$ = valor final de la acci\'on. Halle su funci\'on de probabilidad
  y valor esperado.
\end{statement}

\begin{statement}{d}
  Halle y grafique la funci\'on de distribuci\'on d ela variable aleatoria
  $Y$ = n\'umero de periodos hasta que la acci\'on suba.
\end{statement}

\end{document}