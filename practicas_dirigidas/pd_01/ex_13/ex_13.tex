\documentclass{article}
\usepackage[utf8]{inputenc}
\usepackage{amsfonts,latexsym,amsthm,amssymb,amsmath,amscd,euscript}
\usepackage{mathtools}
\usepackage{framed}
% Descomentar fullpage cuando se quiera utilizar menos margen horizontal
%\usepackage{fullpage}
\usepackage{hyperref}
    \hypersetup{colorlinks=true,citecolor=blue,urlcolor =black,linkbordercolor={1 0 0}}

\newenvironment{statement}[1]{\smallskip\noindent\color[rgb]{1.00,0.00,0.50} {\bf #1.}}{}
\allowdisplaybreaks[1]

% Comandos para teoremas, definiciones, ejemplos, lemas, etc. para sus respectivos body types.
\renewcommand*{\proofname}{Prueba}
\renewcommand{\contentsname}{Contenido}

\newtheorem{theorem}{Teorema}
\newtheorem*{proposition}{Proposici\'on}
\newtheorem{lemma}[theorem]{Lema}
\newtheorem{corollary}[theorem]{Corolario}
\newtheorem{conjecture}[theorem]{Conjetura}
\newtheorem*{postulate}{Postulado}
\theoremstyle{definition}
\newtheorem{defn}[theorem]{Definici\'on}
\newtheorem{example}[theorem]{Ejemplo}

\theoremstyle{remark}
\newtheorem*{remark}{Observaci\'on}
\newtheorem*{notation}{Notaci\'on}
\newtheorem*{note}{Nota}

% Define tus comandos para hacer la vida más fácil.
\newcommand{\BR}{\mathbb R}
\newcommand{\BC}{\mathbb C}
\newcommand{\BF}{\mathbb F}
\newcommand{\BQ}{\mathbb Q}
\newcommand{\BZ}{\mathbb Z}
\newcommand{\BN}{\mathbb N}

\title{EST241 Estad\'istica Inferencial}
\author{Manuel Loaiza Vasquez}
\date{Septiembre 2021}

\begin{document}

\maketitle

\vspace*{-0.25in}
\centerline{Pontificia Universidad Cat\'olica del Per\'u}
\centerline{Lima, Per\'u}
\centerline{\href{mailto:manuel.loaiza@pucp.edu.pe}{{\tt manuel.loaiza@pucp.edu.pe}}}
\vspace*{0.15in}

\begin{framed}
  Soluci\'on del ejercicio $13$ de la pr\'actica dirigida $1$ del curso
  Estad\'istica Inferencial dictado por el profesor Luis Valdivieso Serrano
  y el jefe de pr\'actica Gerald Lozano.
\end{framed}

\begin{statement}{13}
  Las ventas de un producto ocurren de acuerdo a un proceso de Poisson a una
  tasa $\omega = \sqrt{k} / 3$ ventas por d\'ia, donde $k$ es la cantidad de
  soles diarios que el productor gasta en propaganda.
  El producto se vende a $80$ soles la unidad y su costo de producci\'on por
  unidad es $20$ soles. Si cada venta se limita a un producto:
\end{statement}

\begin{statement}{a}
  Para $k = 9$, calcule la probabilidad de que en $5$ d\'ias se tenga una
  utilidad mayor a los $300$ soles.
\end{statement}

\begin{statement}{b}
  Asumamos $k = 9$.
  Suponga que se han vendido en los primeros $10$ d\'ias del mes ya $12$
  productos y que al vendedor encargado se le ha prometido una comisi\'on apenas
  logre vender $15$ productos, ¿qu\'e tiempo se espera transcurra hasta que
  se le d\'e la comisi\'on y qu\'e probabilidad existe de que esta la reciba
  en el mes en curso?
\end{statement}

\begin{statement}{c}
  Para una compa\~na de dos meses, determine el valor \'optimo de $k$ de tal
  manera que la utilidad esperada del productor sea m\'axima.
\end{statement}

\end{document}