\documentclass{article}
\usepackage[utf8]{inputenc}
\usepackage{amsfonts,latexsym,amsthm,amssymb,amsmath,amscd,euscript}
\usepackage{mathtools}
\usepackage{framed}
% Descomentar fullpage cuando se quiera utilizar menos margen horizontal
%\usepackage{fullpage}
\usepackage{hyperref}
    \hypersetup{colorlinks=true,citecolor=blue,urlcolor =black,linkbordercolor={1 0 0}}

\newenvironment{statement}[1]{\smallskip\noindent\color[rgb]{1.00,0.00,0.50} {\bf #1.}}{}
\allowdisplaybreaks[1]

% Comandos para teoremas, definiciones, ejemplos, lemas, etc. para sus respectivos body types.
\renewcommand*{\proofname}{Prueba}
\renewcommand{\contentsname}{Contenido}

\newtheorem{theorem}{Teorema}
\newtheorem*{proposition}{Proposici\'on}
\newtheorem{lemma}[theorem]{Lema}
\newtheorem{corollary}[theorem]{Corolario}
\newtheorem{conjecture}[theorem]{Conjetura}
\newtheorem*{postulate}{Postulado}
\theoremstyle{definition}
\newtheorem{defn}[theorem]{Definici\'on}
\newtheorem{example}[theorem]{Ejemplo}

\theoremstyle{remark}
\newtheorem*{remark}{Observaci\'on}
\newtheorem*{notation}{Notaci\'on}
\newtheorem*{note}{Nota}

% Define tus comandos para hacer la vida más fácil.
\newcommand{\BR}{\mathbb R}
\newcommand{\BC}{\mathbb C}
\newcommand{\BF}{\mathbb F}
\newcommand{\BQ}{\mathbb Q}
\newcommand{\BZ}{\mathbb Z}
\newcommand{\BN}{\mathbb N}

\title{EST241 Estad\'istica Inferencial}
\author{Manuel Loaiza Vasquez}
\date{Septiembre 2021}

\begin{document}

\maketitle

\vspace*{-0.25in}
\centerline{Pontificia Universidad Cat\'olica del Per\'u}
\centerline{Lima, Per\'u}
\centerline{\href{mailto:manuel.loaiza@pucp.edu.pe}{{\tt manuel.loaiza@pucp.edu.pe}}}
\vspace*{0.15in}

\begin{framed}
  Soluci\'on del ejercicio $10$ de la pr\'actica dirigida $1$ del curso
  Estad\'istica Inferencial dictado por el profesor Luis Valdivieso Serrano
  y el jefe de pr\'actica Gerald Lozano.
\end{framed}

\begin{statement}{10}
  En un modelo econ\'omico, el precio unitario de un buen sufre peque\~nas
  perturbaciones aleatorias de modo que se convierte en una variable aleatoria
  continua $X \sim \mathcal{N}(\mu, \sigma^2)$, donde $\mu$ es el precio de
  equilibirio y $\sigma$ mide el margen m\'as probable de variaci\'on alrededor
  de $\mu$. C\'alculos te\'oricos indican que con $97.72\%$ de probabilidad
  el precio se mantendr\'a debajo de las $12$ unidades monetarias y con
  $15.87\%$ de probabilidad estar\'a debajo de las $9$ unidades monetarias.
\end{statement}

\begin{statement}{a}
  Halle el precio de equilibirio $\mu$ y la constante $\sigma$.
\end{statement}

\begin{statement}{b}
  La funci\'on de demanda en este mercado es de $Q(X)$ unidades del bien seg\'un
  $Q(X) = 8000 - 2X$.
  Halle la cantidad esperada de dinero que gastar\'an los consumidores en el
  mercado de este bien.
\end{statement}

\end{document}