\documentclass{article}
\usepackage[utf8]{inputenc}
\usepackage{amsfonts,latexsym,amsthm,amssymb,amsmath,amscd,euscript}
\usepackage{mathtools}
\usepackage{framed}
% Descomentar fullpage cuando se quiera utilizar menos margen horizontal
%\usepackage{fullpage}
\usepackage{hyperref}
    \hypersetup{colorlinks=true,citecolor=blue,urlcolor =black,linkbordercolor={1 0 0}}

\newenvironment{statement}[1]{\smallskip\noindent\color[rgb]{1.00,0.00,0.50} {\bf #1.}}{}
\allowdisplaybreaks[1]

% Comandos para teoremas, definiciones, ejemplos, lemas, etc. para sus respectivos body types.
\renewcommand*{\proofname}{Prueba}
\renewcommand{\contentsname}{Contenido}

\newtheorem{theorem}{Teorema}
\newtheorem*{proposition}{Proposici\'on}
\newtheorem{lemma}[theorem]{Lema}
\newtheorem{corollary}[theorem]{Corolario}
\newtheorem{conjecture}[theorem]{Conjetura}
\newtheorem*{postulate}{Postulado}
\theoremstyle{definition}
\newtheorem{defn}[theorem]{Definici\'on}
\newtheorem{example}[theorem]{Ejemplo}

\theoremstyle{remark}
\newtheorem*{remark}{Observaci\'on}
\newtheorem*{notation}{Notaci\'on}
\newtheorem*{note}{Nota}

% Define tus comandos para hacer la vida más fácil.
\newcommand{\BR}{\mathbb R}
\newcommand{\BC}{\mathbb C}
\newcommand{\BF}{\mathbb F}
\newcommand{\BQ}{\mathbb Q}
\newcommand{\BZ}{\mathbb Z}
\newcommand{\BN}{\mathbb N}

\title{EST241 Estad\'istica Inferencial}
\author{Manuel Loaiza Vasquez}
\date{Septiembre 2021}

\begin{document}

\maketitle

\vspace*{-0.25in}
\centerline{Pontificia Universidad Cat\'olica del Per\'u}
\centerline{Lima, Per\'u}
\centerline{\href{mailto:manuel.loaiza@pucp.edu.pe}{{\tt manuel.loaiza@pucp.edu.pe}}}
\vspace*{0.15in}

\begin{framed}
  Soluci\'on del ejercicio $8$ de la pr\'actica dirigida $1$ del curso
  Estad\'istica Inferencial dictado por el profesor Luis Valdivieso Serrano
  y el jefe de pr\'actica Gerald Lozano.
\end{framed}

\begin{statement}{8}
  En una mina se desea dise\~nar una piscina de oxidaci\'on que procese desechos
  l\'iquidos de una operaci\'on de extracci\'on.
  La piscina debe tener una capacidad de $K\,m^3$ y recibir diariamente entre
  las $8$ am y $9$ am un caudal de desechos de $X\,m^3 / h$, luego de haber
  sido completamente desaguada.
  Esta variable aleatoria $X \sim \Gamma(\alpha,\,\beta = 0.02)$.
  El costo normal de operaci\'on de la piscina ser\'a de $4K$ soles, pero si en
  un d\'ia el caudal que recibe supera a su capacidad, se activar\'a un protocolo
  de emergencia que derivar\'a el resto de los desechos a otro lugar con un costo
  adicional de $12$ soles por cada metro c\'ubico que supere a la capacidad de
  la piscina.
  Suponga que dos economistas $A$ y $B$ han propuesto respectivamente para la
  variable aleatoria continua $X$ los par\'ametros de $\alpha$ igual a $1$ y $4$.
\end{statement}

\begin{statement}{a}
  ¿Para cu\'al de los ingenieros la variabilidad del volumen de desechos que
  alberga diariamente la piscina ser\'a mayor?
\end{statement}

\begin{statement}{b}
  Si $K = 100$ y se tiene un sistema de alerta que se activa cuando la piscina
  supera los $80\,m^3$, ¿para cu\'al de los dos modelos de los ingenieros se
  tendr\'a una mayor probabilidad de que se active la alarma?
\end{statement}

\begin{statement}{c}
  Muestre, indicando los valores de $a$ y $b$, que el costo de operaci\'on
  por d\'ia viene dado por
  \[
    C(X) =
    \begin{cases}
      a K             & \text{si $0 < X \leq K$}\\
      (a - b) K + b X & \text{si $X > K$}.
    \end{cases}  
  \]
\end{statement}

\begin{statement}{d}
  Si $K = 100$ y se asume el modelo del economista $B$, ¿con qu\'e probabilidad
  el costo de operaci\'on por d\'ia superar\'a los $700$ soles?
\end{statement}

\begin{statement}{e}
  Halle el valor \'optimo de $K$, es decir, aquel valor que le permita a la mina
  minimizar su costo esperado de operaci\'on por d\'ia.
  Asuma para ello el modelo del economista $A$.
\end{statement}

\end{document}