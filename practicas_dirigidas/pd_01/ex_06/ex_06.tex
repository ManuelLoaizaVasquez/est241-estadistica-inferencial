\documentclass{article}
\usepackage[utf8]{inputenc}
\usepackage{amsfonts,latexsym,amsthm,amssymb,amsmath,amscd,euscript}
\usepackage{mathtools}
\usepackage{framed}
% Descomentar fullpage cuando se quiera utilizar menos margen horizontal
%\usepackage{fullpage}
\usepackage{hyperref}
    \hypersetup{colorlinks=true,citecolor=blue,urlcolor =black,linkbordercolor={1 0 0}}

\newenvironment{statement}[1]{\smallskip\noindent\color[rgb]{1.00,0.00,0.50} {\bf #1.}}{}
\allowdisplaybreaks[1]

% Comandos para teoremas, definiciones, ejemplos, lemas, etc. para sus respectivos body types.
\renewcommand*{\proofname}{Prueba}
\renewcommand{\contentsname}{Contenido}

\newtheorem{theorem}{Teorema}
\newtheorem*{proposition}{Proposici\'on}
\newtheorem{lemma}[theorem]{Lema}
\newtheorem{corollary}[theorem]{Corolario}
\newtheorem{conjecture}[theorem]{Conjetura}
\newtheorem*{postulate}{Postulado}
\theoremstyle{definition}
\newtheorem{defn}[theorem]{Definici\'on}
\newtheorem{example}[theorem]{Ejemplo}

\theoremstyle{remark}
\newtheorem*{remark}{Observaci\'on}
\newtheorem*{notation}{Notaci\'on}
\newtheorem*{note}{Nota}

% Define tus comandos para hacer la vida más fácil.
\newcommand{\BR}{\mathbb R}
\newcommand{\BC}{\mathbb C}
\newcommand{\BF}{\mathbb F}
\newcommand{\BQ}{\mathbb Q}
\newcommand{\BZ}{\mathbb Z}
\newcommand{\BN}{\mathbb N}

\title{EST241 Estad\'istica Inferencial}
\author{Manuel Loaiza Vasquez}
\date{Septiembre 2021}

\begin{document}

\maketitle

\vspace*{-0.25in}
\centerline{Pontificia Universidad Cat\'olica del Per\'u}
\centerline{Lima, Per\'u}
\centerline{\href{mailto:manuel.loaiza@pucp.edu.pe}{{\tt manuel.loaiza@pucp.edu.pe}}}
\vspace*{0.15in}

\begin{framed}
  Soluci\'on del ejercicio $6$ de la pr\'actica dirigida $1$ del curso
  Estad\'istica Inferencial dictado por el profesor Luis Valdivieso Serrano
  y el jefe de pr\'actica Gerald Lozano.
\end{framed}

\begin{statement}{6}
  Si una m\'aquina de una imprenta se malogra, esta puede destinarse a uno de
  dos talleres $A$ o $B$ para su reparaci\'on.
  El tiempo de reparaci\'on en el taller $A$ es una variable aleatoria
  $X \sim \mathcal{N}(\mu = 5,\,\sigma^2 = 9)$ y el tiempo re reparaci\'on en el
  taller $B$ es una variable aleatoria $Y \sim \mathcal{N}(\mu = 7,\,\sigma^2 = 0.81)$.
\end{statement}

\begin{statement}{a}
  ¿En cu\'al taller es m\'as probable que la m\'aquina se repare en menos de $8$ horas?
\end{statement}

\begin{statement}{b}
  Si cada hora de reparaci\'on de la m\'aquina genera una p\'erdida de $50$
  soles, el taller $A$ cobra por la reparaci\'on de la m\'aquina
  $H_A(X) = 25 + 30 X^2$ y el taller $b$ cobra $H_B(Y) = 10 + 20 Y^2$.
  En t\'erminos de costo esperado, ¿cu\'al taller es m\'as conveniente?
\end{statement}

\end{document}