\documentclass{article}
\usepackage[utf8]{inputenc}
\usepackage{amsfonts,latexsym,amsthm,amssymb,amsmath,amscd,euscript}
\usepackage{mathtools}
\usepackage{framed}
% Descomentar fullpage cuando se quiera utilizar menos margen horizontal
%\usepackage{fullpage}
\usepackage{hyperref}
    \hypersetup{colorlinks=true,citecolor=blue,urlcolor =black,linkbordercolor={1 0 0}}

\newenvironment{statement}[1]{\smallskip\noindent\color[rgb]{1.00,0.00,0.50} {\bf #1.}}{}
\allowdisplaybreaks[1]

% Comandos para teoremas, definiciones, ejemplos, lemas, etc. para sus respectivos body types.
\renewcommand*{\proofname}{Prueba}
\renewcommand{\contentsname}{Contenido}

\newtheorem{theorem}{Teorema}
\newtheorem*{proposition}{Proposici\'on}
\newtheorem{lemma}[theorem]{Lema}
\newtheorem{corollary}[theorem]{Corolario}
\newtheorem{conjecture}[theorem]{Conjetura}
\newtheorem*{postulate}{Postulado}
\theoremstyle{definition}
\newtheorem{defn}[theorem]{Definici\'on}
\newtheorem{example}[theorem]{Ejemplo}

\theoremstyle{remark}
\newtheorem*{remark}{Observaci\'on}
\newtheorem*{notation}{Notaci\'on}
\newtheorem*{note}{Nota}

% Define tus comandos para hacer la vida más fácil.
\newcommand{\BR}{\mathbb R}
\newcommand{\BC}{\mathbb C}
\newcommand{\BF}{\mathbb F}
\newcommand{\BQ}{\mathbb Q}
\newcommand{\BZ}{\mathbb Z}
\newcommand{\BN}{\mathbb N}

\title{EST241 Estad\'istica Inferencial}
\author{Manuel Loaiza Vasquez}
\date{Septiembre 2021}

\begin{document}

\maketitle

\vspace*{-0.25in}
\centerline{Pontificia Universidad Cat\'olica del Per\'u}
\centerline{Lima, Per\'u}
\centerline{\href{mailto:manuel.loaiza@pucp.edu.pe}{{\tt manuel.loaiza@pucp.edu.pe}}}
\vspace*{0.15in}

\begin{framed}
  Soluci\'on del ejercicio $9$ de la pr\'actica dirigida $1$ del curso
  Estad\'istica Inferencial dictado por el profesor Luis Valdivieso Serrano
  y el jefe de pr\'actica Gerald Lozano.
\end{framed}

\begin{statement}{9}
  Una persona tiene una operaci\'on financiera con probabilidad $p$ de ser
  exitosa cada vez que la realiza. La persona estima que la probabilidad de
  tener \'exito es la mitad de la probabilidad de no tenerlo.
\end{statement}

\begin{statement}{a}
  Halle $p$.
\end{statement}

\begin{statement}{b}
  Sea $X$ = n\'umero de operaciones exitosas sobre un total de $n$ operaciones,
  con $n = 6$. Identifique la distribuci\'on de $X$ y halle la probabilidad de
  que tenga el doble de \'exitos que de fracasos.
\end{statement}

\begin{statement}{c}
  Suponga que cada operaci\'on exitosa reporta una ganancia de $4\%$ y si no
  lo es implica una p\'erdida de $1\%$. La persona realiz\'o $6$ veces esta
  operaci\'on, poniendo un capital inicial de $1000$ soles y sin retirar dinero.
  Halle la probabilidad de que termina con menos dinero del que puso.
\end{statement}

\begin{statement}{d}
  Si en el inciso (c) la persona retira un monto de $100$ soles con probabilidad
  $0.8$ cada vez que la operaci\'on resulta exitosa, ¿cu\'al ser\'ia el
  capital que se esperar\'ia obtenga al t\'ermino de las $6$ operaciones?
\end{statement}

\end{document}