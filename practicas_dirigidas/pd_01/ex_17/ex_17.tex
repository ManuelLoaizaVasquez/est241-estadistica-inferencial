\documentclass{article}
\usepackage[utf8]{inputenc}
\usepackage{amsfonts,latexsym,amsthm,amssymb,amsmath,amscd,euscript}
\usepackage{mathtools}
\usepackage{framed}
% Descomentar fullpage cuando se quiera utilizar menos margen horizontal
%\usepackage{fullpage}
\usepackage{hyperref}
    \hypersetup{colorlinks=true,citecolor=blue,urlcolor =black,linkbordercolor={1 0 0}}

\newenvironment{statement}[1]{\smallskip\noindent\color[rgb]{1.00,0.00,0.50} {\bf #1.}}{}
\allowdisplaybreaks[1]

% Comandos para teoremas, definiciones, ejemplos, lemas, etc. para sus respectivos body types.
\renewcommand*{\proofname}{Prueba}
\renewcommand{\contentsname}{Contenido}

\newtheorem{theorem}{Teorema}
\newtheorem*{proposition}{Proposici\'on}
\newtheorem{lemma}[theorem]{Lema}
\newtheorem{corollary}[theorem]{Corolario}
\newtheorem{conjecture}[theorem]{Conjetura}
\newtheorem*{postulate}{Postulado}
\theoremstyle{definition}
\newtheorem{defn}[theorem]{Definici\'on}
\newtheorem{example}[theorem]{Ejemplo}

\theoremstyle{remark}
\newtheorem*{remark}{Observaci\'on}
\newtheorem*{notation}{Notaci\'on}
\newtheorem*{note}{Nota}

% Define tus comandos para hacer la vida más fácil.
\newcommand{\BR}{\mathbb R}
\newcommand{\BC}{\mathbb C}
\newcommand{\BF}{\mathbb F}
\newcommand{\BQ}{\mathbb Q}
\newcommand{\BZ}{\mathbb Z}
\newcommand{\BN}{\mathbb N}

\title{EST241 Estad\'istica Inferencial}
\author{Manuel Loaiza Vasquez}
\date{Septiembre 2021}

\begin{document}

\maketitle

\vspace*{-0.25in}
\centerline{Pontificia Universidad Cat\'olica del Per\'u}
\centerline{Lima, Per\'u}
\centerline{\href{mailto:manuel.loaiza@pucp.edu.pe}{{\tt manuel.loaiza@pucp.edu.pe}}}
\vspace*{0.15in}

\begin{framed}
  Soluci\'on del ejercicio $0$ de la pr\'actica dirigida $1$ del curso
  Estad\'istica Inferencial dictado por el profesor Luis Valdivieso Serrano
  y el jefe de pr\'actica Gerald Lozano.
\end{framed}

\begin{statement}{17}
  El ingreso mensual (en miles de d\'olares) que un inversionista recibe por las
  acciones que posee, es una variable aleatoria continua
  $X \sim \text{Lognormal}(\mu = 1, \sigma^2 = 1)$.
\end{statement}

\begin{statement}{a}
  El inversionista tiene una deuda de $5000$ y espera pagarla con lo que recibe
  a fin de mes por su inversi\'on. ¿Con qu\'e probabilidad podr\'a pagar su
  deuda el inversionista?
\end{statement}

\begin{statement}{b}
  El inversionista gasta $400$ mensuales en manejar su inversi\'on y una
  administradora de valores le ofrece manejar sus acciones cobr\'andole $1.2\%$
  de los ingresos que le genere, m\'as una cantidad fija de $10$ por gastos generales.
  ¿Qu\'e le conviene m\'as al inversionista?
\end{statement}

\end{document}