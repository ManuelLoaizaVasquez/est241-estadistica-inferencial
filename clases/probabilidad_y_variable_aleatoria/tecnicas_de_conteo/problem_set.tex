\documentclass{article}
\usepackage[utf8]{inputenc}
\usepackage{amsfonts,latexsym,amsthm,amssymb,amsmath,amscd,euscript}
\usepackage{mathtools}
\usepackage{framed}
% Descomentar fullpage cuando se quiera utilizar menos margen horizontal
%\usepackage{fullpage}
\usepackage{hyperref}
    \hypersetup{colorlinks=true,citecolor=blue,urlcolor =black,linkbordercolor={1 0 0}}

\newenvironment{statement}[1]{\smallskip\noindent\color[rgb]{1.00,0.00,0.50} {\bf #1.}}{}
\allowdisplaybreaks[1]

% Comandos para teoremas, definiciones, ejemplos, lemas, etc. para sus respectivos body types.
\renewcommand*{\proofname}{Prueba}
\renewcommand{\contentsname}{Contenido}

\newtheorem{theorem}{Teorema}
\newtheorem*{proposition}{Proposici\'on}
\newtheorem{lemma}[theorem]{Lema}
\newtheorem{corollary}[theorem]{Corolario}
\newtheorem{conjecture}[theorem]{Conjetura}
\newtheorem*{postulate}{Postulado}
\theoremstyle{definition}
\newtheorem{defn}[theorem]{Definici\'on}
\newtheorem{example}[theorem]{Ejemplo}

\theoremstyle{remark}
\newtheorem*{remark}{Observaci\'on}
\newtheorem*{notation}{Notaci\'on}
\newtheorem*{note}{Nota}

% Define tus comandos para hacer la vida más fácil.
\newcommand{\BR}{\mathbb R}
\newcommand{\BC}{\mathbb C}
\newcommand{\BF}{\mathbb F}
\newcommand{\BQ}{\mathbb Q}
\newcommand{\BZ}{\mathbb Z}
\newcommand{\BN}{\mathbb N}

\title{EST241 Estad\'istica Inferencial}
\author{Manuel Loaiza Vasquez}
\date{Septiembre 2021}

\begin{document}

\maketitle

\vspace*{-0.25in}
\centerline{Pontificia Universidad Cat\'olica del Per\'u}
\centerline{Lima, Per\'u}
\centerline{\href{mailto:manuel.loaiza@pucp.edu.pe}{{\tt manuel.loaiza@pucp.edu.pe}}}
\vspace*{0.15in}

\begin{framed}
  Lista de ejercicios de combinatoria preparada por Manuel Loaiza para calentar
  antes de la pr\'actica calificada.
\end{framed}

\begin{statement}{1}
  Pruebe que
  \[
      \binom{n + m}{r} = \binom{n}{0}\binom{m}{r} +
      \binom{n}{1}\binom{m}{r - 1} +
      \cdots +
      \binom{n}{r}\binom{m}{0}.
  \]
\end{statement}

\begin{statement}{2}
  Pruebe la identidad
  \[
    \binom{n}{k} = \sum_{i = k}^n \binom{i - 1}{k - 1},\, n \geq k.  
  \]
\end{statement}

\documentclass{article}
\usepackage[utf8]{inputenc}
\usepackage{amsfonts,latexsym,amsthm,amssymb,amsmath,amscd,euscript}
\usepackage{mathtools}
\usepackage{framed}
% Descomentar fullpage cuando se quiera utilizar menos margen horizontal
%\usepackage{fullpage}
\usepackage{hyperref}
    \hypersetup{colorlinks=true,citecolor=blue,urlcolor =black,linkbordercolor={1 0 0}}

\newenvironment{statement}[1]{\smallskip\noindent\color[rgb]{1.00,0.00,0.50} {\bf #1.}}{}
\allowdisplaybreaks[1]

% Comandos para teoremas, definiciones, ejemplos, lemas, etc. para sus respectivos body types.
\renewcommand*{\proofname}{Prueba}
\renewcommand{\contentsname}{Contenido}

\newtheorem{theorem}{Teorema}
\newtheorem*{proposition}{Proposici\'on}
\newtheorem{lemma}[theorem]{Lema}
\newtheorem{corollary}[theorem]{Corolario}
\newtheorem{conjecture}[theorem]{Conjetura}
\newtheorem*{postulate}{Postulado}
\theoremstyle{definition}
\newtheorem{defn}[theorem]{Definici\'on}
\newtheorem{example}[theorem]{Ejemplo}

\theoremstyle{remark}
\newtheorem*{remark}{Observaci\'on}
\newtheorem*{notation}{Notaci\'on}
\newtheorem*{note}{Nota}

% Define tus comandos para hacer la vida más fácil.
\newcommand{\BR}{\mathbb R}
\newcommand{\BC}{\mathbb C}
\newcommand{\BF}{\mathbb F}
\newcommand{\BQ}{\mathbb Q}
\newcommand{\BZ}{\mathbb Z}
\newcommand{\BN}{\mathbb N}

\title{EST241 Estad\'istica Inferencial}
\author{Manuel Loaiza Vasquez}
\date{Septiembre 2021}

\begin{document}

\maketitle

\vspace*{-0.25in}
\centerline{Pontificia Universidad Cat\'olica del Per\'u}
\centerline{Lima, Per\'u}
\centerline{\href{mailto:manuel.loaiza@pucp.edu.pe}{{\tt manuel.loaiza@pucp.edu.pe}}}
\vspace*{0.15in}

\begin{framed}
  Soluci\'on del ejercicio $3$ de la pr\'actica dirigida $1$ del curso
  Estad\'istica Inferencial dictado por el profesor Luis Valdivieso Serrano
  y el jefe de pr\'actica Gerald Lozano.
\end{framed}

\begin{statement}{3}
  Un comerciante mayorista suele comprar $S$ unidades de un bien a precio unitario
  de $3$ soles, para revenderlo luego a $7$ soles la unidad durante la temporada
  de ventas. Pasada la \'epoca de ventas, el sobrante se pierde.
  La cantidad de producto que le pueden demandar a este mayorista es una
  variable aleatoria continua $X$ con funci\'on de densidad exponencial
  $X \sim \text{Exp}(\beta = 0.01)$.
\end{statement}

\begin{statement}{a}
  Halle la f\'ormula que define la utilidad $U(X, S)$ como funci\'on de la
  variable aleatoria $X$ y el stock no aleatorio $S$.
\end{statement}

\begin{statement}{b}
  El comerciante desea saber cu\'al es el stock \'optimo $S$ de producto que
  debiera comprar para su negocio, de modo que la utilidad esperada
  $\mathbb{E}[U(X, S)]$ sea m\'axima. ¿Qu\'e valor de $S$ recomendar\'ia?
\end{statement}

\end{document}

\documentclass{article}
\usepackage[utf8]{inputenc}
\usepackage{amsfonts,latexsym,amsthm,amssymb,amsmath,amscd,euscript}
\usepackage{mathtools}
\usepackage{framed}
% Descomentar fullpage cuando se quiera utilizar menos margen horizontal
%\usepackage{fullpage}
\usepackage{hyperref}
    \hypersetup{colorlinks=true,citecolor=blue,urlcolor =black,linkbordercolor={1 0 0}}

\newenvironment{statement}[1]{\smallskip\noindent\color[rgb]{1.00,0.00,0.50} {\bf #1.}}{}
\allowdisplaybreaks[1]

% Comandos para teoremas, definiciones, ejemplos, lemas, etc. para sus respectivos body types.
\renewcommand*{\proofname}{Prueba}
\renewcommand{\contentsname}{Contenido}

\newtheorem{theorem}{Teorema}
\newtheorem*{proposition}{Proposici\'on}
\newtheorem{lemma}[theorem]{Lema}
\newtheorem{corollary}[theorem]{Corolario}
\newtheorem{conjecture}[theorem]{Conjetura}
\newtheorem*{postulate}{Postulado}
\theoremstyle{definition}
\newtheorem{defn}[theorem]{Definici\'on}
\newtheorem{example}[theorem]{Ejemplo}

\theoremstyle{remark}
\newtheorem*{remark}{Observaci\'on}
\newtheorem*{notation}{Notaci\'on}
\newtheorem*{note}{Nota}

% Define tus comandos para hacer la vida más fácil.
\newcommand{\BR}{\mathbb R}
\newcommand{\BC}{\mathbb C}
\newcommand{\BF}{\mathbb F}
\newcommand{\BQ}{\mathbb Q}
\newcommand{\BZ}{\mathbb Z}
\newcommand{\BN}{\mathbb N}

\title{EST241 Estad\'istica Inferencial}
\author{Manuel Loaiza Vasquez}
\date{Septiembre 2021}

\begin{document}

\maketitle

\vspace*{-0.25in}
\centerline{Pontificia Universidad Cat\'olica del Per\'u}
\centerline{Lima, Per\'u}
\centerline{\href{mailto:manuel.loaiza@pucp.edu.pe}{{\tt manuel.loaiza@pucp.edu.pe}}}
\vspace*{0.15in}

\begin{framed}
  Soluci\'on del ejercicio $4$ de la pr\'actica dirigida $1$ del curso
  Estad\'istica Inferencial dictado por el profesor Luis Valdivieso Serrano
  y el jefe de pr\'actica Gerald Lozano.
\end{framed}

\begin{statement}{4}
  Si la rentabilidad mensual $X$ de una inversi\'on es una variable aleatoria
  continua con rango $R_X = [-1, 8]$ y para la funci\'on de densidad
  $f_X(x)$ se tienen las siguientes dos situaciones distintas, que genrar dos
  modelos para las frecuencias de $X$.
\end{statement}

\begin{statement}{a}
  Halle $f_X$ si se sabe que con $1 / 9$ de probabilidad de rentabilidad puede
  ser negativa.
\end{statement}

\begin{statement}{b}
  Halle $\mathbb{E}[X]$ en cada modelo e interprete. De acuerdo a su interpretaci\'on,
  ¿vale la pena invertir en esta acci\'on en alguna de las situaciones?
\end{statement}

\end{document}

\documentclass{article}
\usepackage[utf8]{inputenc}
\usepackage{amsfonts,latexsym,amsthm,amssymb,amsmath,amscd,euscript}
\usepackage{mathtools}
\usepackage{framed}
% Descomentar fullpage cuando se quiera utilizar menos margen horizontal
%\usepackage{fullpage}
\usepackage{hyperref}
    \hypersetup{colorlinks=true,citecolor=blue,urlcolor =black,linkbordercolor={1 0 0}}

\newenvironment{statement}[1]{\smallskip\noindent\color[rgb]{1.00,0.00,0.50} {\bf #1.}}{}
\allowdisplaybreaks[1]

% Comandos para teoremas, definiciones, ejemplos, lemas, etc. para sus respectivos body types.
\renewcommand*{\proofname}{Prueba}
\renewcommand{\contentsname}{Contenido}

\newtheorem{theorem}{Teorema}
\newtheorem*{proposition}{Proposici\'on}
\newtheorem{lemma}[theorem]{Lema}
\newtheorem{corollary}[theorem]{Corolario}
\newtheorem{conjecture}[theorem]{Conjetura}
\newtheorem*{postulate}{Postulado}
\theoremstyle{definition}
\newtheorem{defn}[theorem]{Definici\'on}
\newtheorem{example}[theorem]{Ejemplo}

\theoremstyle{remark}
\newtheorem*{remark}{Observaci\'on}
\newtheorem*{notation}{Notaci\'on}
\newtheorem*{note}{Nota}

% Define tus comandos para hacer la vida más fácil.
\newcommand{\BR}{\mathbb R}
\newcommand{\BC}{\mathbb C}
\newcommand{\BF}{\mathbb F}
\newcommand{\BQ}{\mathbb Q}
\newcommand{\BZ}{\mathbb Z}
\newcommand{\BN}{\mathbb N}

\title{EST241 Estad\'istica Inferencial}
\author{Manuel Loaiza Vasquez}
\date{Septiembre 2021}

\begin{document}

\maketitle

\vspace*{-0.25in}
\centerline{Pontificia Universidad Cat\'olica del Per\'u}
\centerline{Lima, Per\'u}
\centerline{\href{mailto:manuel.loaiza@pucp.edu.pe}{{\tt manuel.loaiza@pucp.edu.pe}}}
\vspace*{0.15in}

\begin{framed}
  Soluci\'on del ejercicio $5$ de la pr\'actica dirigida $1$ del curso
  Estad\'istica Inferencial dictado por el profesor Luis Valdivieso Serrano
  y el jefe de pr\'actica Gerald Lozano.
\end{framed}

\begin{statement}{5}
  Una acci\'on valuada incialmente en S$/. 10$, se cotiza en dos periodos.
  Durante cada periodo la acci\'on puede bajar, permanecer igual o subir en S$/. 1$.
  Las probabilidades de estos eventos en el primer periodo son $0.2$, $0.5$ y $0.3$ respectivamente.
  En el segundo, la acci\'on puede volver a subir con probabilidad $0.3$ y bajar
  despu\'es de subir en el primero con probabilidad $0.1$.
  Ahora, si la acci\'on se mantiene igual en el periodo inicial, puede subir en
  el segundo con probabilidad $0.4$ y puede mantenerse igual con probabilidad
  $0.35$. Finalmente, si la acci\'on baja en el primer periodo, lo seguir\'a
  haciendo en el segundo periodo con probabilidad $0.5$, y subir\'a con
  probabilidad $0.1$.
\end{statement}

\begin{statement}{a}
  Halle la probabilidad de que el valor final de la acci\'on sea de S$/. 12$.
\end{statement}

\begin{statement}{b}
  Si el valor final de la acci\'on iguala al valor inicial, ¿cu\'al ser\'ia
  la probabilidad de que la acci\'on haya bajado?
\end{statement}

\begin{statement}{c}
  Sea $X$ = valor final de la acci\'on. Halle su funci\'on de probabilidad
  y valor esperado.
\end{statement}

\begin{statement}{d}
  Halle y grafique la funci\'on de distribuci\'on d ela variable aleatoria
  $Y$ = n\'umero de periodos hasta que la acci\'on suba.
\end{statement}

\end{document}

\documentclass{article}
\usepackage[utf8]{inputenc}
\usepackage{amsfonts,latexsym,amsthm,amssymb,amsmath,amscd,euscript}
\usepackage{mathtools}
\usepackage{framed}
% Descomentar fullpage cuando se quiera utilizar menos margen horizontal
%\usepackage{fullpage}
\usepackage{hyperref}
    \hypersetup{colorlinks=true,citecolor=blue,urlcolor =black,linkbordercolor={1 0 0}}

\newenvironment{statement}[1]{\smallskip\noindent\color[rgb]{1.00,0.00,0.50} {\bf #1.}}{}
\allowdisplaybreaks[1]

% Comandos para teoremas, definiciones, ejemplos, lemas, etc. para sus respectivos body types.
\renewcommand*{\proofname}{Prueba}
\renewcommand{\contentsname}{Contenido}

\newtheorem{theorem}{Teorema}
\newtheorem*{proposition}{Proposici\'on}
\newtheorem{lemma}[theorem]{Lema}
\newtheorem{corollary}[theorem]{Corolario}
\newtheorem{conjecture}[theorem]{Conjetura}
\newtheorem*{postulate}{Postulado}
\theoremstyle{definition}
\newtheorem{defn}[theorem]{Definici\'on}
\newtheorem{example}[theorem]{Ejemplo}

\theoremstyle{remark}
\newtheorem*{remark}{Observaci\'on}
\newtheorem*{notation}{Notaci\'on}
\newtheorem*{note}{Nota}

% Define tus comandos para hacer la vida más fácil.
\newcommand{\BR}{\mathbb R}
\newcommand{\BC}{\mathbb C}
\newcommand{\BF}{\mathbb F}
\newcommand{\BQ}{\mathbb Q}
\newcommand{\BZ}{\mathbb Z}
\newcommand{\BN}{\mathbb N}

\title{EST241 Estad\'istica Inferencial}
\author{Manuel Loaiza Vasquez}
\date{Septiembre 2021}

\begin{document}

\maketitle

\vspace*{-0.25in}
\centerline{Pontificia Universidad Cat\'olica del Per\'u}
\centerline{Lima, Per\'u}
\centerline{\href{mailto:manuel.loaiza@pucp.edu.pe}{{\tt manuel.loaiza@pucp.edu.pe}}}
\vspace*{0.15in}

\begin{framed}
  Soluci\'on del ejercicio $6$ de la pr\'actica dirigida $1$ del curso
  Estad\'istica Inferencial dictado por el profesor Luis Valdivieso Serrano
  y el jefe de pr\'actica Gerald Lozano.
\end{framed}

\begin{statement}{6}
  Si una m\'aquina de una imprenta se malogra, esta puede destinarse a uno de
  dos talleres $A$ o $B$ para su reparaci\'on.
  El tiempo de reparaci\'on en el taller $A$ es una variable aleatoria
  $X \sim \mathcal{N}(\mu = 5,\,\sigma^2 = 9)$ y el tiempo re reparaci\'on en el
  taller $B$ es una variable aleatoria $Y \sim \mathcal{N}(\mu = 7,\,\sigma^2 = 0.81)$.
\end{statement}

\begin{statement}{a}
  ¿En cu\'al taller es m\'as probable que la m\'aquina se repare en menos de $8$ horas?
\end{statement}

\begin{statement}{b}
  Si cada hora de reparaci\'on de la m\'aquina genera una p\'erdida de $50$
  soles, el taller $A$ cobra por la reparaci\'on de la m\'aquina
  $H_A(X) = 25 + 30 X^2$ y el taller $b$ cobra $H_B(Y) = 10 + 20 Y^2$.
  En t\'erminos de costo esperado, ¿cu\'al taller es m\'as conveniente?
\end{statement}

\end{document}

\documentclass{article}
\usepackage[utf8]{inputenc}
\usepackage{amsfonts,latexsym,amsthm,amssymb,amsmath,amscd,euscript}
\usepackage{mathtools}
\usepackage{framed}
% Descomentar fullpage cuando se quiera utilizar menos margen horizontal
%\usepackage{fullpage}
\usepackage{hyperref}
    \hypersetup{colorlinks=true,citecolor=blue,urlcolor =black,linkbordercolor={1 0 0}}

\newenvironment{statement}[1]{\smallskip\noindent\color[rgb]{1.00,0.00,0.50} {\bf #1.}}{}
\allowdisplaybreaks[1]

% Comandos para teoremas, definiciones, ejemplos, lemas, etc. para sus respectivos body types.
\renewcommand*{\proofname}{Prueba}
\renewcommand{\contentsname}{Contenido}

\newtheorem{theorem}{Teorema}
\newtheorem*{proposition}{Proposici\'on}
\newtheorem{lemma}[theorem]{Lema}
\newtheorem{corollary}[theorem]{Corolario}
\newtheorem{conjecture}[theorem]{Conjetura}
\newtheorem*{postulate}{Postulado}
\theoremstyle{definition}
\newtheorem{defn}[theorem]{Definici\'on}
\newtheorem{example}[theorem]{Ejemplo}

\theoremstyle{remark}
\newtheorem*{remark}{Observaci\'on}
\newtheorem*{notation}{Notaci\'on}
\newtheorem*{note}{Nota}

% Define tus comandos para hacer la vida más fácil.
\newcommand{\BR}{\mathbb R}
\newcommand{\BC}{\mathbb C}
\newcommand{\BF}{\mathbb F}
\newcommand{\BQ}{\mathbb Q}
\newcommand{\BZ}{\mathbb Z}
\newcommand{\BN}{\mathbb N}

\title{EST241 Estad\'istica Inferencial}
\author{Manuel Loaiza Vasquez}
\date{Septiembre 2021}

\begin{document}

\maketitle

\vspace*{-0.25in}
\centerline{Pontificia Universidad Cat\'olica del Per\'u}
\centerline{Lima, Per\'u}
\centerline{\href{mailto:manuel.loaiza@pucp.edu.pe}{{\tt manuel.loaiza@pucp.edu.pe}}}
\vspace*{0.15in}

\begin{framed}
  Soluci\'on del ejercicio $7$ de la pr\'actica dirigida $1$ del curso
  Estad\'istica Inferencial dictado por el profesor Luis Valdivieso Serrano
  y el jefe de pr\'actica Gerald Lozano.
\end{framed}

\begin{statement}{7}
  Una persona maneja con tres inversiones $A, B$ y $C$, estimando que las
  probabilidades de tener utilidades son $0.2$, $0.7$ y $0.5$ respectivamente.
\end{statement}

\begin{statement}{a}
  Si hay independencia entre las inversiones, calcule la probabilidad de que logre
  utilidad en todas las inversiones.
\end{statement}

\begin{statement}{b}
  Si la probabilidad de lograr utilidades en $A$ y $B$ es $0.15$, calcule
  la probabilidad de que no logre utilidades en $B$ dado que s\'i las obtuvo $A$.
\end{statement}

\begin{statement}{c}
  Si la probabilidad de lograr utilidades en $A$ y $B$ es $0.15$ y $C$ es
  independiente de las otras dos inversiones. Calcule la probabilidad de que se
  logre alguna utilidad con la cartera.
\end{statement}

\end{document}

\begin{statement}{8}
  Un torneo todos contra todos de $n$ participantes es un torneo en el cual
  cada una de las $\binom{n}{2}$ parejas de participantes juega uno contra el otro
  exactamente una vez, con un resultado de cualquier juego obteniendo un
  participante ganador y otro perdedor.
  Sea $k$ un entero fijo, $k < n$, una pregunta que nos puede interesar es si
  es que es posible que el resultado del torneo sea tal que, para todo conjunto
  de $k$ jugadores, existe un jugador que puede vencer a cada integrante de ese
  conjunto. Pruebe que si
  \[
    \binom{n}{k}\left[1 - \left(\frac{1}{2}\right)^k\right]^{n - k} < 1
  \]
  entonces dicho resultado es posible.
\end{statement}

\begin{statement}{9}
  Dados dos n\'umeros naturales $n$ y $k$. Hallar la m\'axima potencia de $k$
  que divide a $n!$.
\end{statement}

\documentclass{article}
\usepackage[utf8]{inputenc}
\usepackage{amsfonts,latexsym,amsthm,amssymb,amsmath,amscd,euscript}
\usepackage{mathtools}
\usepackage{framed}
% Descomentar fullpage cuando se quiera utilizar menos margen horizontal
%\usepackage{fullpage}
\usepackage{hyperref}
    \hypersetup{colorlinks=true,citecolor=blue,urlcolor =black,linkbordercolor={1 0 0}}

\newenvironment{statement}[1]{\smallskip\noindent\color[rgb]{1.00,0.00,0.50} {\bf #1.}}{}
\allowdisplaybreaks[1]

% Comandos para teoremas, definiciones, ejemplos, lemas, etc. para sus respectivos body types.
\renewcommand*{\proofname}{Prueba}
\renewcommand{\contentsname}{Contenido}

\newtheorem{theorem}{Teorema}
\newtheorem*{proposition}{Proposici\'on}
\newtheorem{lemma}[theorem]{Lema}
\newtheorem{corollary}[theorem]{Corolario}
\newtheorem{conjecture}[theorem]{Conjetura}
\newtheorem*{postulate}{Postulado}
\theoremstyle{definition}
\newtheorem{defn}[theorem]{Definici\'on}
\newtheorem{example}[theorem]{Ejemplo}

\theoremstyle{remark}
\newtheorem*{remark}{Observaci\'on}
\newtheorem*{notation}{Notaci\'on}
\newtheorem*{note}{Nota}

% Define tus comandos para hacer la vida más fácil.
\newcommand{\BR}{\mathbb R}
\newcommand{\BC}{\mathbb C}
\newcommand{\BF}{\mathbb F}
\newcommand{\BQ}{\mathbb Q}
\newcommand{\BZ}{\mathbb Z}
\newcommand{\BN}{\mathbb N}

\title{EST241 Estad\'istica Inferencial}
\author{Manuel Loaiza Vasquez}
\date{Septiembre 2021}

\begin{document}

\maketitle

\vspace*{-0.25in}
\centerline{Pontificia Universidad Cat\'olica del Per\'u}
\centerline{Lima, Per\'u}
\centerline{\href{mailto:manuel.loaiza@pucp.edu.pe}{{\tt manuel.loaiza@pucp.edu.pe}}}
\vspace*{0.15in}

\begin{framed}
  Soluci\'on del ejercicio $10$ de la pr\'actica dirigida $1$ del curso
  Estad\'istica Inferencial dictado por el profesor Luis Valdivieso Serrano
  y el jefe de pr\'actica Gerald Lozano.
\end{framed}

\begin{statement}{10}
  En un modelo econ\'omico, el precio unitario de un buen sufre peque\~nas
  perturbaciones aleatorias de modo que se convierte en una variable aleatoria
  continua $X \sim \mathcal{N}(\mu, \sigma^2)$, donde $\mu$ es el precio de
  equilibirio y $\sigma$ mide el margen m\'as probable de variaci\'on alrededor
  de $\mu$. C\'alculos te\'oricos indican que con $97.72\%$ de probabilidad
  el precio se mantendr\'a debajo de las $12$ unidades monetarias y con
  $15.87\%$ de probabilidad estar\'a debajo de las $9$ unidades monetarias.
\end{statement}

\begin{statement}{a}
  Halle el precio de equilibirio $\mu$ y la constante $\sigma$.
\end{statement}

\begin{statement}{b}
  La funci\'on de demanda en este mercado es de $Q(X)$ unidades del bien seg\'un
  $Q(X) = 8000 - 2X$.
  Halle la cantidad esperada de dinero que gastar\'an los consumidores en el
  mercado de este bien.
\end{statement}

\end{document}

\end{document}
