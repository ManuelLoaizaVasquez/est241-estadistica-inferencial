\documentclass{article}
\usepackage[utf8]{inputenc}
\usepackage{amsfonts,latexsym,amsthm,amssymb,amsmath,amscd,euscript}
\usepackage{mathtools}
\usepackage{framed}
% Descomentar fullpage cuando se quiera utilizar menos margen horizontal
%\usepackage{fullpage}
\usepackage{hyperref}
    \hypersetup{colorlinks=true,citecolor=blue,urlcolor =black,linkbordercolor={1 0 0}}

\newenvironment{statement}[1]{\smallskip\noindent\color[rgb]{1.00,0.00,0.50} {\bf #1.}}{}
\allowdisplaybreaks[1]

% Comandos para teoremas, definiciones, ejemplos, lemas, etc. para sus respectivos body types.
\renewcommand*{\proofname}{Prueba}
\renewcommand{\contentsname}{Contenido}

\newtheorem{theorem}{Teorema}
\newtheorem*{proposition}{Proposici\'on}
\newtheorem{lemma}[theorem]{Lema}
\newtheorem{corollary}[theorem]{Corolario}
\newtheorem{conjecture}[theorem]{Conjetura}
\newtheorem*{postulate}{Postulado}
\theoremstyle{definition}
\newtheorem{defn}[theorem]{Definici\'on}
\newtheorem{example}[theorem]{Ejemplo}

\theoremstyle{remark}
\newtheorem*{remark}{Observaci\'on}
\newtheorem*{notation}{Notaci\'on}
\newtheorem*{note}{Nota}

% Define tus comandos para hacer la vida más fácil.
\newcommand{\BR}{\mathbb R}
\newcommand{\BC}{\mathbb C}
\newcommand{\BF}{\mathbb F}
\newcommand{\BQ}{\mathbb Q}
\newcommand{\BZ}{\mathbb Z}
\newcommand{\BN}{\mathbb N}

\title{EST241 Estad\'istica Inferencial}
\author{Manuel Loaiza Vasquez}
\date{Septiembre 2021}

\begin{document}

\maketitle

\vspace*{-0.25in}
\centerline{Pontificia Universidad Cat\'olica del Per\'u}
\centerline{Lima, Per\'u}
\centerline{\href{mailto:manuel.loaiza@pucp.edu.pe}{{\tt manuel.loaiza@pucp.edu.pe}}}
\vspace*{0.15in}

\begin{framed}
  Lista de ejercicios de combinatoria preparada por Manuel Loaiza para calentar
  antes de la pr\'actica calificada.
\end{framed}

\begin{statement}{1}
  Pruebe que
  \[
      \binom{n + m}{r} = \binom{n}{0}\binom{m}{r} +
      \binom{n}{1}\binom{m}{r - 1} +
      \cdots +
      \binom{n}{r}\binom{m}{0}.
  \]
\end{statement}

\begin{statement}{2}
  Pruebe la identidad
  \[
    \binom{n}{k} = \sum_{i = k}^n \binom{i - 1}{k - 1},\, n \geq k.  
  \]
\end{statement}

\begin{statement}{3}
  Pruebe que
  \[
    \binom{n}{2} = \binom{k}{2} + k (n - k) + \binom{n - k}{2},\, 1 \leq k \leq n.  
  \]
\end{statement}

\documentclass{article}
\usepackage[utf8]{inputenc}
\usepackage{amsfonts,latexsym,amsthm,amssymb,amsmath,amscd,euscript}
\usepackage{mathtools}
\usepackage{framed}
% Descomentar fullpage cuando se quiera utilizar menos margen horizontal
%\usepackage{fullpage}
\usepackage{hyperref}
    \hypersetup{colorlinks=true,citecolor=blue,urlcolor =black,linkbordercolor={1 0 0}}

\newenvironment{statement}[1]{\smallskip\noindent\color[rgb]{1.00,0.00,0.50} {\bf #1.}}{}
\allowdisplaybreaks[1]

% Comandos para teoremas, definiciones, ejemplos, lemas, etc. para sus respectivos body types.
\renewcommand*{\proofname}{Prueba}
\renewcommand{\contentsname}{Contenido}

\newtheorem{theorem}{Teorema}
\newtheorem*{proposition}{Proposici\'on}
\newtheorem{lemma}[theorem]{Lema}
\newtheorem{corollary}[theorem]{Corolario}
\newtheorem{conjecture}[theorem]{Conjetura}
\newtheorem*{postulate}{Postulado}
\theoremstyle{definition}
\newtheorem{defn}[theorem]{Definici\'on}
\newtheorem{example}[theorem]{Ejemplo}

\theoremstyle{remark}
\newtheorem*{remark}{Observaci\'on}
\newtheorem*{notation}{Notaci\'on}
\newtheorem*{note}{Nota}

% Define tus comandos para hacer la vida más fácil.
\newcommand{\BR}{\mathbb R}
\newcommand{\BC}{\mathbb C}
\newcommand{\BF}{\mathbb F}
\newcommand{\BQ}{\mathbb Q}
\newcommand{\BZ}{\mathbb Z}
\newcommand{\BN}{\mathbb N}

\title{EST241 Estad\'istica Inferencial}
\author{Manuel Loaiza Vasquez}
\date{Septiembre 2021}

\begin{document}

\maketitle

\vspace*{-0.25in}
\centerline{Pontificia Universidad Cat\'olica del Per\'u}
\centerline{Lima, Per\'u}
\centerline{\href{mailto:manuel.loaiza@pucp.edu.pe}{{\tt manuel.loaiza@pucp.edu.pe}}}
\vspace*{0.15in}

\begin{framed}
  Soluci\'on del ejercicio $4$ de la pr\'actica dirigida $1$ del curso
  Estad\'istica Inferencial dictado por el profesor Luis Valdivieso Serrano
  y el jefe de pr\'actica Gerald Lozano.
\end{framed}

\begin{statement}{4}
  Si la rentabilidad mensual $X$ de una inversi\'on es una variable aleatoria
  continua con rango $R_X = [-1, 8]$ y para la funci\'on de densidad
  $f_X(x)$ se tienen las siguientes dos situaciones distintas, que genrar dos
  modelos para las frecuencias de $X$.
\end{statement}

\begin{statement}{a}
  Halle $f_X$ si se sabe que con $1 / 9$ de probabilidad de rentabilidad puede
  ser negativa.
\end{statement}

\begin{statement}{b}
  Halle $\mathbb{E}[X]$ en cada modelo e interprete. De acuerdo a su interpretaci\'on,
  ¿vale la pena invertir en esta acci\'on en alguna de las situaciones?
\end{statement}

\end{document}

\documentclass{article}
\usepackage[utf8]{inputenc}
\usepackage{amsfonts,latexsym,amsthm,amssymb,amsmath,amscd,euscript}
\usepackage{mathtools}
\usepackage{framed}
% Descomentar fullpage cuando se quiera utilizar menos margen horizontal
%\usepackage{fullpage}
\usepackage{hyperref}
    \hypersetup{colorlinks=true,citecolor=blue,urlcolor =black,linkbordercolor={1 0 0}}

\newenvironment{statement}[1]{\smallskip\noindent\color[rgb]{1.00,0.00,0.50} {\bf #1.}}{}
\allowdisplaybreaks[1]

% Comandos para teoremas, definiciones, ejemplos, lemas, etc. para sus respectivos body types.
\renewcommand*{\proofname}{Prueba}
\renewcommand{\contentsname}{Contenido}

\newtheorem{theorem}{Teorema}
\newtheorem*{proposition}{Proposici\'on}
\newtheorem{lemma}[theorem]{Lema}
\newtheorem{corollary}[theorem]{Corolario}
\newtheorem{conjecture}[theorem]{Conjetura}
\newtheorem*{postulate}{Postulado}
\theoremstyle{definition}
\newtheorem{defn}[theorem]{Definici\'on}
\newtheorem{example}[theorem]{Ejemplo}

\theoremstyle{remark}
\newtheorem*{remark}{Observaci\'on}
\newtheorem*{notation}{Notaci\'on}
\newtheorem*{note}{Nota}

% Define tus comandos para hacer la vida más fácil.
\newcommand{\BR}{\mathbb R}
\newcommand{\BC}{\mathbb C}
\newcommand{\BF}{\mathbb F}
\newcommand{\BQ}{\mathbb Q}
\newcommand{\BZ}{\mathbb Z}
\newcommand{\BN}{\mathbb N}

\title{EST241 Estad\'istica Inferencial}
\author{Manuel Loaiza Vasquez}
\date{Septiembre 2021}

\begin{document}

\maketitle

\vspace*{-0.25in}
\centerline{Pontificia Universidad Cat\'olica del Per\'u}
\centerline{Lima, Per\'u}
\centerline{\href{mailto:manuel.loaiza@pucp.edu.pe}{{\tt manuel.loaiza@pucp.edu.pe}}}
\vspace*{0.15in}

\begin{framed}
  Soluci\'on del ejercicio $5$ de la pr\'actica dirigida $1$ del curso
  Estad\'istica Inferencial dictado por el profesor Luis Valdivieso Serrano
  y el jefe de pr\'actica Gerald Lozano.
\end{framed}

\begin{statement}{5}
  Una acci\'on valuada incialmente en S$/. 10$, se cotiza en dos periodos.
  Durante cada periodo la acci\'on puede bajar, permanecer igual o subir en S$/. 1$.
  Las probabilidades de estos eventos en el primer periodo son $0.2$, $0.5$ y $0.3$ respectivamente.
  En el segundo, la acci\'on puede volver a subir con probabilidad $0.3$ y bajar
  despu\'es de subir en el primero con probabilidad $0.1$.
  Ahora, si la acci\'on se mantiene igual en el periodo inicial, puede subir en
  el segundo con probabilidad $0.4$ y puede mantenerse igual con probabilidad
  $0.35$. Finalmente, si la acci\'on baja en el primer periodo, lo seguir\'a
  haciendo en el segundo periodo con probabilidad $0.5$, y subir\'a con
  probabilidad $0.1$.
\end{statement}

\begin{statement}{a}
  Halle la probabilidad de que el valor final de la acci\'on sea de S$/. 12$.
\end{statement}

\begin{statement}{b}
  Si el valor final de la acci\'on iguala al valor inicial, ¿cu\'al ser\'ia
  la probabilidad de que la acci\'on haya bajado?
\end{statement}

\begin{statement}{c}
  Sea $X$ = valor final de la acci\'on. Halle su funci\'on de probabilidad
  y valor esperado.
\end{statement}

\begin{statement}{d}
  Halle y grafique la funci\'on de distribuci\'on d ela variable aleatoria
  $Y$ = n\'umero de periodos hasta que la acci\'on suba.
\end{statement}

\end{document}

\begin{statement}{6}
  Pruebe que
  \[
    \sum_{k = 0}^n \binom{n}{k} 2^k = 3^n.  
  \]
\end{statement}

\documentclass{article}
\usepackage[utf8]{inputenc}
\usepackage{amsfonts,latexsym,amsthm,amssymb,amsmath,amscd,euscript}
\usepackage{mathtools}
\usepackage{framed}
% Descomentar fullpage cuando se quiera utilizar menos margen horizontal
%\usepackage{fullpage}
\usepackage{hyperref}
    \hypersetup{colorlinks=true,citecolor=blue,urlcolor =black,linkbordercolor={1 0 0}}

\newenvironment{statement}[1]{\smallskip\noindent\color[rgb]{1.00,0.00,0.50} {\bf #1.}}{}
\allowdisplaybreaks[1]

% Comandos para teoremas, definiciones, ejemplos, lemas, etc. para sus respectivos body types.
\renewcommand*{\proofname}{Prueba}
\renewcommand{\contentsname}{Contenido}

\newtheorem{theorem}{Teorema}
\newtheorem*{proposition}{Proposici\'on}
\newtheorem{lemma}[theorem]{Lema}
\newtheorem{corollary}[theorem]{Corolario}
\newtheorem{conjecture}[theorem]{Conjetura}
\newtheorem*{postulate}{Postulado}
\theoremstyle{definition}
\newtheorem{defn}[theorem]{Definici\'on}
\newtheorem{example}[theorem]{Ejemplo}

\theoremstyle{remark}
\newtheorem*{remark}{Observaci\'on}
\newtheorem*{notation}{Notaci\'on}
\newtheorem*{note}{Nota}

% Define tus comandos para hacer la vida más fácil.
\newcommand{\BR}{\mathbb R}
\newcommand{\BC}{\mathbb C}
\newcommand{\BF}{\mathbb F}
\newcommand{\BQ}{\mathbb Q}
\newcommand{\BZ}{\mathbb Z}
\newcommand{\BN}{\mathbb N}

\title{EST241 Estad\'istica Inferencial}
\author{Manuel Loaiza Vasquez}
\date{Septiembre 2021}

\begin{document}

\maketitle

\vspace*{-0.25in}
\centerline{Pontificia Universidad Cat\'olica del Per\'u}
\centerline{Lima, Per\'u}
\centerline{\href{mailto:manuel.loaiza@pucp.edu.pe}{{\tt manuel.loaiza@pucp.edu.pe}}}
\vspace*{0.15in}

\begin{framed}
  Soluci\'on del ejercicio $7$ de la pr\'actica dirigida $1$ del curso
  Estad\'istica Inferencial dictado por el profesor Luis Valdivieso Serrano
  y el jefe de pr\'actica Gerald Lozano.
\end{framed}

\begin{statement}{7}
  Una persona maneja con tres inversiones $A, B$ y $C$, estimando que las
  probabilidades de tener utilidades son $0.2$, $0.7$ y $0.5$ respectivamente.
\end{statement}

\begin{statement}{a}
  Si hay independencia entre las inversiones, calcule la probabilidad de que logre
  utilidad en todas las inversiones.
\end{statement}

\begin{statement}{b}
  Si la probabilidad de lograr utilidades en $A$ y $B$ es $0.15$, calcule
  la probabilidad de que no logre utilidades en $B$ dado que s\'i las obtuvo $A$.
\end{statement}

\begin{statement}{c}
  Si la probabilidad de lograr utilidades en $A$ y $B$ es $0.15$ y $C$ es
  independiente de las otras dos inversiones. Calcule la probabilidad de que se
  logre alguna utilidad con la cartera.
\end{statement}

\end{document}

\documentclass{article}
\usepackage[utf8]{inputenc}
\usepackage{amsfonts,latexsym,amsthm,amssymb,amsmath,amscd,euscript}
\usepackage{mathtools}
\usepackage{framed}
% Descomentar fullpage cuando se quiera utilizar menos margen horizontal
%\usepackage{fullpage}
\usepackage{hyperref}
    \hypersetup{colorlinks=true,citecolor=blue,urlcolor =black,linkbordercolor={1 0 0}}

\newenvironment{statement}[1]{\smallskip\noindent\color[rgb]{1.00,0.00,0.50} {\bf #1.}}{}
\allowdisplaybreaks[1]

% Comandos para teoremas, definiciones, ejemplos, lemas, etc. para sus respectivos body types.
\renewcommand*{\proofname}{Prueba}
\renewcommand{\contentsname}{Contenido}

\newtheorem{theorem}{Teorema}
\newtheorem*{proposition}{Proposici\'on}
\newtheorem{lemma}[theorem]{Lema}
\newtheorem{corollary}[theorem]{Corolario}
\newtheorem{conjecture}[theorem]{Conjetura}
\newtheorem*{postulate}{Postulado}
\theoremstyle{definition}
\newtheorem{defn}[theorem]{Definici\'on}
\newtheorem{example}[theorem]{Ejemplo}

\theoremstyle{remark}
\newtheorem*{remark}{Observaci\'on}
\newtheorem*{notation}{Notaci\'on}
\newtheorem*{note}{Nota}

% Define tus comandos para hacer la vida más fácil.
\newcommand{\BR}{\mathbb R}
\newcommand{\BC}{\mathbb C}
\newcommand{\BF}{\mathbb F}
\newcommand{\BQ}{\mathbb Q}
\newcommand{\BZ}{\mathbb Z}
\newcommand{\BN}{\mathbb N}

\title{EST241 Estad\'istica Inferencial}
\author{Manuel Loaiza Vasquez}
\date{Septiembre 2021}

\begin{document}

\maketitle

\vspace*{-0.25in}
\centerline{Pontificia Universidad Cat\'olica del Per\'u}
\centerline{Lima, Per\'u}
\centerline{\href{mailto:manuel.loaiza@pucp.edu.pe}{{\tt manuel.loaiza@pucp.edu.pe}}}
\vspace*{0.15in}

\begin{framed}
  Soluci\'on del ejercicio $8$ de la pr\'actica dirigida $1$ del curso
  Estad\'istica Inferencial dictado por el profesor Luis Valdivieso Serrano
  y el jefe de pr\'actica Gerald Lozano.
\end{framed}

\begin{statement}{8}
  En una mina se desea dise\~nar una piscina de oxidaci\'on que procese desechos
  l\'iquidos de una operaci\'on de extracci\'on.
  La piscina debe tener una capacidad de $K\,m^3$ y recibir diariamente entre
  las $8$ am y $9$ am un caudal de desechos de $X\,m^3 / h$, luego de haber
  sido completamente desaguada.
  Esta variable aleatoria $X \sim \Gamma(\alpha,\,\beta = 0.02)$.
  El costo normal de operaci\'on de la piscina ser\'a de $4K$ soles, pero si en
  un d\'ia el caudal que recibe supera a su capacidad, se activar\'a un protocolo
  de emergencia que derivar\'a el resto de los desechos a otro lugar con un costo
  adicional de $12$ soles por cada metro c\'ubico que supere a la capacidad de
  la piscina.
  Suponga que dos economistas $A$ y $B$ han propuesto respectivamente para la
  variable aleatoria continua $X$ los par\'ametros de $\alpha$ igual a $1$ y $4$.
\end{statement}

\begin{statement}{a}
  ¿Para cu\'al de los ingenieros la variabilidad del volumen de desechos que
  alberga diariamente la piscina ser\'a mayor?
\end{statement}

\begin{statement}{b}
  Si $K = 100$ y se tiene un sistema de alerta que se activa cuando la piscina
  supera los $80\,m^3$, ¿para cu\'al de los dos modelos de los ingenieros se
  tendr\'a una mayor probabilidad de que se active la alarma?
\end{statement}

\begin{statement}{c}
  Muestre, indicando los valores de $a$ y $b$, que el costo de operaci\'on
  por d\'ia viene dado por
  \[
    C(X) =
    \begin{cases}
      a K             & \text{si $0 < X \leq K$}\\
      (a - b) K + b X & \text{si $X > K$}.
    \end{cases}  
  \]
\end{statement}

\begin{statement}{d}
  Si $K = 100$ y se asume el modelo del economista $B$, ¿con qu\'e probabilidad
  el costo de operaci\'on por d\'ia superar\'a los $700$ soles?
\end{statement}

\begin{statement}{e}
  Halle el valor \'optimo de $K$, es decir, aquel valor que le permita a la mina
  minimizar su costo esperado de operaci\'on por d\'ia.
  Asuma para ello el modelo del economista $A$.
\end{statement}

\end{document}

\begin{statement}{9}
  Dados dos n\'umeros naturales $n$ y $k$. Hallar la m\'axima potencia de $k$
  que divide a $n!$.
\end{statement}

\begin{statement}{10}
  Calcule el n\'umero de maneras de colocar $k$ alfiles en un tablero de ajedrez
  de $n \times n$ tal que ning\'un par de alfil se puede atacar mutuamente.
\end{statement}

\end{document}
